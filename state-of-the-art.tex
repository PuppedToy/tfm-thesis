Existen numerosos marcos de trabajo para todo tipo de lenguajes de programación. La mayor parte de estos marcos producen esqueletos rápidos para ponerse a trabajar con un conjunto de librerías en poco tiempo. Todos ellos tienen su utilidad y forman parte el día a día de muchos desarrolladores.

En cuanto a calidad de código, existen también numerosas herramientas que facilitan las tareas. Existen analizadores estáticos de código, herramientas de integración contínua, marcos de trabajo para pruebas que incluyen el uso de imágenes instantáneas automáticas y existen generadores automáticos de código, que producen esqueletos de componentes específicos en el momento en que se necesitan.

Los desarrolladores, a lo largo de su carrera, van probando unas tecnologías y otras para ir decidiendo cuáles usar en un proyecto nuevo y, con práctica, hilar todas estas herramientas en un solo paquete se vuelve una tarea relativamente rápida. Sin embargo, cada proyecto tiene unas necesidades y no siempre sale barato el esfuerzo de unificar todas estas herramientas que nos facilitan la vida.

En este trabajo se ha dedicado tiempo a entender qué tecnologías se utilizan y de qué forma en el mundo del desarrollo web y, como sucede normalmente, existen todas las herramientas mencionadas por separado, pero no existe ningún marco que las unifique de forma que el código de lanzamiento que hay que escribir sea mínimo. Gracias al soporte de las herramientas a ficheros estáticos de configuración, con un simple catálogo de tecnologías, un usuario podría llegar a configurar en proyecto bajo un enfoque holístico sin escribir una sola línea de código. Ese marco de trabajo, esa herramienta de unificación, es precisamente el hueco en el mercado que voy a intentar explotar con este proyecto.
