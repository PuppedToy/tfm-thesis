\textbf{INTRODUCCIÓN}: las pruebas de concepto destinadas a ser rápidas generalmente se convierten en productos a gran escala en poco tiempo. Si no se cuida la calidad desde el principio, el proyecto puede ser difícil de mantener y, por lo tanto, puede fallar o ser demasiado costoso.

\textbf{OBJETIVOS}: El objetivo principal es crear un marco de trabajo capaz de suavizar la carga de crear una prueba de concepto de calidad, manteniendo buenas prácticas y pruebas automatizadas desde el principio. Este marco tendrá como alcance el desarrollo web.

\textbf{MÉTODOS}: El nuevo marco de trabajo se creó para hacer frente a las tediosas tareas de configuración antes de que comience el proyecto, teniendo en cuenta la integración continua, las pruebas, las buenas prácticas y la flexibilidad con las tecnologías disponibles. Este marco tiene un esqueleto preparado para garantizar la calidad. Para probar que este marco es útil, se ha creado un proyecto llamado \textit{TODO List} utilizando métodos tradicionales y el nuevo marco simultáneamente, comparando el tiempo que costó construir las dos versiones del proyecto.

\textbf{RESULTADOS}: El proyecto tradicional se creó más rápido que la nueva versión del marco, ya que las pruebas fueron la tarea que más tiempo requirió. Sin embargo, el proyecto tradicional incluyó 20 errores que el nuevo marco evitó debido a todas las pruebas realizadas en él. Resulta que depurar y resolver esos errores hizo que el proyecto tradicional llevara tanto tiempo como el nuevo marco de trabajo. Al final, el nuevo marco proporcionó toda la ayuda de configuración necesaria para aumentar la velocidad de producción para que coincida con los proyectos sin calidad, manteniendo una buena batería de pruebas y buenas prácticas.

\textbf{CONCLUSIÓN}: Por lo general, vale la pena crear un código de calidad desde el principio y este nuevo marco de trabajo ayudará a proporcionar toda la configuración necesaria.
