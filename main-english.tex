% !TeX spellcheck = en-US
% !TeX encoding = utf8
% !TeX program = lualatex
% !BIB program = biber
% -*- coding:utf-8 mod:LaTeX -*-


% vv  scroll down to line 200 for content  vv


\let\ifdeutsch\iffalse
\let\ifenglisch\iftrue
\input{pre-documentclass}
\documentclass[
  % fontsize=11pt is the standard
  a4paper,  % Standard format - only KOMAScript uses paper=a4 - https://tex.stackexchange.com/a/61044/9075
  twoside,  % we are optimizing for both screen and two-side printing. So the page numbers will jump, but the content is configured to stay in the middle (by using the geometry package)
  bibliography=totoc,
  %               idxtotoc,   %Index ins Inhaltsverzeichnis
  %               liststotoc, %List of X ins Inhaltsverzeichnis, mit liststotocnumbered werden die Abbildungsverzeichnisse nummeriert
  headsepline,
  cleardoublepage=empty,
  parskip=half,
  %               draft    % um zu sehen, wo noch nachgebessert werden muss - wichtig, da Bindungskorrektur mit drin
  draft=false
]{scrbook}
\input{config}


\usepackage[
  title={Marco de trabajo para la definición de tecnologías aplicables a un proyecto software},
  author={Alejandro Alcázar},
  type={Trabajo de Fin de Máster},
  institute=Universidad Politécnica de Madrid, % or other institute names - or just a plain string using {Demo\\Demo...}
  course={Máster Universitario en Ingeniería Informática},
  examiner={Óscar Dieste},
  supervisor={Óscar Dieste},
  startdate={January 27, 2020},
  enddate={TBD}
]{scientific-thesis-cover}

\newacronym{ssr}{SSR}{Server-Side Rendering}
\newacronym{cms}{CMS}{Content Management System}
\newacronym{seo}{SEO}{Search Engine Optimization}
\newacronym{e2e}{e2e}{End To End}
\newacronym{orm}{ORM}{Object-Relational Mapping}
\newacronym{sql}{SQL}{Structured Query Language}
\newacronym{dod}{DOD}{Document-Oriented Database}
\newacronym{bson}{BSON}{Binary JSON}
\newacronym{json}{JSON}{JavaScript Object Notation}
\newacronym{ddns}{DDNS}{Dynamic Domain Name System}
\newacronym{api}{API}{Application Program Interface}


\makeindex

\begin{document}

%tex4ht-Konvertierung verschönern
\iftex4ht
  % tell tex4ht to create picures also for formulas starting with '$'
  % WARNING: a tex4ht run now takes forever!
  \Configure{$}{\PicMath}{\EndPicMath}{}
  %$ % <- syntax highlighting fix for emacs
  \Css{body {text-align:justify;}}

  %conversion of .pdf to .png
  \Configure{graphics*}
  {pdf}
  {\Needs{"convert \csname Gin@base\endcsname.pdf
      \csname Gin@base\endcsname.png"}%
    \Picture[pict]{\csname Gin@base\endcsname.png}%
  }
\fi

%\VerbatimFootnotes %verbatim text in Fußnoten erlauben. Geht normalerweise nicht.

\input{commands}
\pagenumbering{arabic}
\Titelblatt

%Eigener Seitenstil fuer die Kurzfassung und das Inhaltsverzeichnis
\deftripstyle{preamble}{}{}{}{}{}{\pagemark}
%Doku zu deftripstyle: scrguide.pdf
\pagestyle{preamble}
\renewcommand*{\chapterpagestyle}{preamble}

\section*{Resumen}

<Short summary of the thesis>

\clearpage

\section*{Summary}

<Translated summry of the thesis>

\cleardoublepage


% BEGIN: Verzeichnisse

\iftex4ht
\else
  \microtypesetup{protrusion=false}
\fi

%%%
% Literaturverzeichnis ins TOC mit aufnehmen, aber nur wenn nichts anderes mehr hilft!
% \addcontentsline{toc}{chapter}{Literaturverzeichnis}
%
% oder zB
%\addcontentsline{toc}{section}{Abkürzungsverzeichnis}
%
%%%

%Produce table of contents
%
%In case you have trouble with headings reaching into the page numbers, enable the following three lines.
%Hint by http://golatex.de/inhaltsverzeichnis-schreibt-ueber-rand-t3106.html
%
%\makeatletter
%\renewcommand{\@pnumwidth}{2em}
%\makeatother
%
\tableofcontents

% Bei einem ungünstigen Seitenumbruch im Inhaltsverzeichnis, kann dieser mit
% \addtocontents{toc}{\protect\newpage}
% an der passenden Stelle im Fließtext erzwungen werden.

\listoffigures
\listoftables

%Wird nur bei Verwendung von der lstlisting-Umgebung mit dem "caption"-Parameter benoetigt
%\lstlistoflistings 
%ansonsten:
\ifdeutsch
  \listof{Listing}{Verzeichnis der Listings}
\else
  \listof{Listing}{List of Listings}
\fi

%mittels \newfloat wurde die Algorithmus-Gleitumgebung definiert.
%Mit folgendem Befehl werden alle floats dieses Typs ausgegeben
\ifdeutsch
  \listof{Algorithmus}{Verzeichnis der Algorithmen}
\else
  \listof{Algorithmus}{List of Algorithms}
\fi
%\listofalgorithms %Ist nur für Algorithmen, die mittels \begin{algorithm} umschlossen werden, nötig

% Abkürzungsverzeichnis
\printnoidxglossaries

\iftex4ht
\else
  %Optischen Randausgleich und Grauwertkorrektur wieder aktivieren
  \microtypesetup{protrusion=true}
\fi

% END: Verzeichnisse


% Headline and footline
\renewcommand*{\chapterpagestyle}{scrplain}
\pagestyle{scrheadings}
\pagestyle{scrheadings}
\ihead[]{}
\chead[]{}
\ohead[]{\headmark}
\cfoot[]{}
\ofoot[\usekomafont{pagenumber}\thepage]{\usekomafont{pagenumber}\thepage}
\ifoot[]{}


%% vv  scroll down for content  vv %%































%%%%%%%%%%%%%%%%%%%%%%%%%%%%%%%%%%%%%%%%%%%%%%%%%%%%%%%%%%%%%%%%%%%%%%%%%%%%%%
%
% Main content starts here
%
%%%%%%%%%%%%%%%%%%%%%%%%%%%%%%%%%%%%%%%%%%%%%%%%%%%%%%%%%%%%%%%%%%%%%%%%%%%%%%


\chapter{Introduction}

This thesis tarts with \cref{chap:k2}.

Lorem ipsum dolor sit amet, consectetur adipiscing elit. Integer eu efficitur dolor. Etiam pellentesque dictum diam, ac ultrices sem hendrerit id. Donec tristique fermentum massa, et viverra elit scelerisque sed. Sed purus odio, tempus a suscipit at, rhoncus sed justo. Proin viverra, lacus nec bibendum fringilla, risus justo iaculis urna, eget molestie arcu mi et purus. Nulla eget nibh eu mauris tincidunt porttitor. Aenean eu lacus scelerisque, faucibus lorem vel, laoreet dolor. Nam sed ultricies magna, eu lacinia enim. Interdum et malesuada fames ac ante ipsum primis in faucibus. Vestibulum posuere tellus sed ex interdum rhoncus. Duis eu ante dolor. In at elit scelerisque, varius mi sit amet, eleifend nisl. Integer metus nulla, viverra vitae justo in, tincidunt ornare nibh. Phasellus feugiat euismod risus vel lobortis. Praesent eu ornare erat, eget congue ante. Donec accumsan, arcu sed porta efficitur, diam massa venenatis sem, quis ultrices risus purus et turpis.

Duis gravida diam quis est congue, vel blandit eros laoreet. Maecenas facilisis scelerisque iaculis. Donec sollicitudin fringilla turpis gravida maximus. Aliquam lacinia malesuada porttitor. Quisque vitae enim id lorem molestie venenatis quis nec turpis. Integer molestie auctor felis, commodo sollicitudin ligula suscipit nec. Suspendisse facilisis venenatis lectus pretium venenatis. Vestibulum eu accumsan libero. Duis consectetur tristique mi. Aliquam nunc felis, volutpat quis sapien vel, maximus porta sem. Morbi pellentesque arcu facilisis massa posuere pulvinar. Aliquam lobortis nisi gravida, interdum leo posuere, sagittis leo. Nulla aliquam mauris ipsum, eu bibendum sem interdum non. Mauris erat orci, sollicitudin ut elit id, tempus viverra nulla. Curabitur sit amet tincidunt orci. Nullam in dolor quam.

Nunc eu vulputate risus. Suspendisse vel ex blandit, viverra erat a, dapibus elit. Etiam blandit lorem sit amet lectus tempor, ac bibendum odio aliquet. Etiam quis magna eu justo sagittis bibendum. Nam elit dolor, tristique ut lobortis sed, cursus ut sapien. Sed malesuada mauris ut lectus condimentum, sit amet tempus erat luctus. Lorem ipsum dolor sit amet, consectetur adipiscing elit. Suspendisse potenti. Vestibulum mollis purus nec blandit venenatis. Vestibulum porttitor hendrerit elit, consequat accumsan libero pharetra sit amet. Cras ante sem, hendrerit eget odio id, ultricies auctor sapien. Etiam iaculis tortor augue, ac sagittis enim elementum eget. Suspendisse lorem urna, posuere eu rhoncus ornare, aliquam eu tellus. Proin sed laoreet orci. Proin accumsan et mi at tincidunt. Nulla facilisis sed sem aliquam facilisis.

We can also typeset \verb|<text>verbatim text</text>|.
Backticks are also rendered correctly: \verb|`words in backticks`|.

\chapter{Chapter Two}
\label{chap:k2}

LaTeX hints are provided in \cref{chap:latexhints}.

\chapter{Estudio de mercado de las tecnologías disponibles}
\label{chap:techMarketResearch}

\section{Entornos de trabajo del lado de cliente}

En esta sección se van a evaluar los entornos de desarrollo más populares en el lado del cliente: React, Svelte, VueJS, Preact y Angular. Como preámbulo, se muestra en la figura \cref{fig:stjs2019:frontend} la satisfacción de los usuarios con respecto a cada entorno. Esto es relevante porque la satisfacción de los usuarios depende de factores como la eficiencia de código y la curva de aprendizaje. Como queremos que nuestro marco de trabajo esté destinado a aplicaciones pequeñas que tienen potencial para crecer, es importante conocer cuál es la opinión media de los usuarios de 2019. Además queremos que el entorno sea accesible al mayor número de usuarios, por lo que la popularidad es el factor más relevante en este estudio.

\begin{figure}
	\centering
	\includegraphics[width=\textwidth]{front_end_frameworks_experience_ranking.png}
	\caption{2019 - Opinión popular de los entornos de trabajo front-end}
	\label{fig:stjs2019:frontend}
\end{figure}

Parece que los dos más importantes en este sentido son React y VueJS (Svelte lo voy a descartar por el poco tiempo en el mercado). Según la comparación de rendimiento que presenta \citet{RWC2019} en su artículo, Vue parece líder en aplicaciones ligeras y rápidas. Por otro lado, en la comparativa que desarrolla \citet{TJSF2019}, se puede apreciar que React es utilizado por un 64.8\% de los desarrolladores web contra los 28.8\% que están con Vue. Además, React cuenta con más soporte y documentación que respaldan sus años de presencia en el mercado, mientras que la para floja de Vue hoy por hoy es su soporte.

Hay otro análisis a tener en cuenta para realizar esta elección, y es la facilidad para depurar el código. La limpieza, la cantidad de líneas y las herramientas de pruebas disponibles cumplemn un papel clave en este trabajo. Para valorarlo, he tenido en cuenta un artículo de \citet{RVVCTOG} que hace la comparativa desde el punto de vista de la calidad de código.

\begin{enumerate}
	\item Tipado: Sabemos que Javascript no es un lenguaje tipado. Sin embargo, cuando hablamos de pruebas, es importante comprobar el tipo de los datos que fluyen a través de la aplicación. Tanto React como Vue permiten comprobaciones de tipos mediante Javascript tipado: TypeScript es una solución global al problema. Sin embargo, es más sencillo usar TypeScript en React. Además, React dispone de una herramienta oficial, Flow, que permite hacer estas comprobaciones de forma todavía más fácil.
	\item Modularidad: Ambos entornos están basados en modularidad de código. Todo depende de lo bien diseñada que esté la aplicación.
	\item Curva de aprendizaje: Vue es líder en este aspecto. Uno de los problemas de React es que su curva de aprendizaje es menos suave que el de otros entornos. Vue está hecho para ser dominado en poco tiempo.
	\item Pruebas y depuración: React es líder en este aspecto. Hay varias herramientas de pruebas, siendo Jest la recomendada por Facebook. Además, React cuenta con una extensión Chrome para depurar sus componentes de forma rápida y sencilla. Aun así, Vue también tiene herramientas de pruebas.
	\item \gls{ssr}: Es importante mejorar la velocidad de descarga y potenciar el \gls{seo}. Para eso, se suele utilizar la técnica de \gls{ssr}. Se puede leer más sobre este tema en el artículo de \citet{SSREXP}. Para este aspecto, ambos entornos tienen sus herramientas. React tiene ahora NextJS y para Vue existe Nuxt.js.
	\item Mantenibilidad del código: Este es uno de los aspectos más importantes, dado que la idea es que el entorno sirva para proyectos pequeños con potencial de crecimiento. El entorno pretende hacer que un proyecto pequeño sea mantenible de forma sencilla, así que este punto es clave. Para este apartado, \citet{RVVMNTB} explica en su artículo que, en sus años de experiencia, React es más mantenible y que leer código de React que han escrito otros es más sencillo. Esto no deja de ser un punto opinable, pero dentro de lo opinable que es, la opinión más extendida aboga para React en aplicaciones más grandes.
\end{enumerate}

Así que, pese a que Vue ha estado creciendo y parece el líder en rendimiento, React parece una elección muy sólida y equilibrada que va a llegar a más desarrolladores y realizar aplicaciones más mantenibles durante el 2020. Por tanto, he decidido realizar el marco de trabajo en React. Si el framework tiene éxito, se planteará extenderlo a otros entornos de trabajo.

Esto tiene repercusiones sobre el esto de toma de decisiones. La figura \cref{fig:react-learnthrough} intenta ilustrar la cantidad de tecnologías que un desarrollador de React debería conocer para dominar todo su potencial. Aun así, esta figura no es más que un mapa conceptual, dado que durante el resto de capítulos se ha ido haciendo una investigación individual de cada tecnología. Aun así, el resultado de las tecnologías en el lado del cliente ha sido bastante parecida a lo que se puede observar en la imagen, así que la incluyo de referencia.

\begin{figure}
  \centering
  \includegraphics[width=\textwidth]{react_learnthrough.png}
  \caption{2020 - El camino del desarrollador de React en 2020}
  \label{fig:react-learnthrough}
\end{figure}



\section{Entornos de trabajo del lado de servidor}

En esta sección se van a evaluar los entornos de desarrollo más populares en el lado del servidor teniendo en cuenta que se va a utilizar React. Si siguiésemos el mismo criterio que con el apartado del lado de cliente, encontraríamos que numerosas fuentes -por ejemplo, \cite{BKETPF1} y \cite{BKETPF2}- están de acuerdo en que PHP es la tecnología más utilizada en el lado de servidor con diferencia. Se puede apreciar en la figura \cref{fig:similartech:backend} sacada de \cite{BKETPF2} en Febrero de 2020. Sin embargo, en este caso particular y por la tecnología elegida en el lado del cliente, se van a valorar positivamente tecnologías que permitan desarrollar al mismo tiempo el cliente y el servidor. PHP es muy utilizado por varias razones:

\begin{enumeration}
	\item Es el más antiguo de los lenguajes de lado de servidor.
	\item Tiene la mayor comunidad que puede tener un lenguaje de servidor y, por tanto, una extensa documentación.
	\item Es utilizado por marcos aun más grandes, como Symphony o Wordpress. Estos son CMS y están fuera del alcance de este proyecto.
\end{enumeration}

Sin embargo, PHP está pensado para directamente renderizar HTML, que es exactamente lo que hace React. Pese a que queremos dar soporte a las tecnologías más extendidas, también queremos estar actualizados con pilas tecnológicas actualizadas y comprobadas. Una de estas pilas tecnológicas de las que hablo es MERN (MongoDB, ExpressJS, ReactJS y NodeJS). Estas pilas tecnológicas están recomendadas por los siguientes motivos:

\begin{enumeration}
	\item Solo hay que aprender un lenguaje, Javascript, lo cual reduce la curva de aprendizaje.
	\item La gestión de datos internos de la aplicación es muy similar a la gestión de base de datos porque MongoDB trabaja con objetos Javascript.
	\item Se pueden hacer contenedores Docker que contengan la pila completa de forma muy sencilla, distribuyendo la carga y haciendo la aplicación escalable. [Faltan citas de todo esto].
\end{enumeration}

Para esto se van a evaluar dos opciones: Express y Next.js. Se ha considerado utilizar múltiples entornos de servidor, entre ellos los que podemos encontrar en la figura \cref{fig:stjs2019:backend}. Sin embargo, teniendo decidido el lado del cliente, la mayoría de opciones no merecen la pena. React trabaja de forma estupenda con NodeJS y lo más habitual es que vayan de la mano. 

\begin{figure}
	\centering
	\includegraphics[width=\textwidth]{similartech_backend_trend_top_10K.png}
	\caption{Leading Framework technologies share on the web - Top 10K Sites}
	\label{fig:similartech:backend}
\end{figure}

\begin{figure}
	\centering
	\includegraphics[width=\textwidth]{back_end_frameworks_experience_ranking.png}
	\caption{2019 - Opinión popular de los entornos de trabajo back-end}
	\label{fig:stjs2019:backend}
\end{figure}

\section{Control de versiones}

El control de versiones consiste en tener todas las versiones de código que se han generado etiquetadas, de forma que se pueda acceder a cualquier versión generada en cualquier momento. Esta es una base imprescindible para un proyecto de calidad, ya que permite trazar errores con mucha facilidad y es una salvaguarda ante fallos. Además, el control de versiones permite tener varias ramas de trabajo, separando el código de la rama de producción, la rama de desarrollo y las distintas ramas de los equipos y características que se están trabjando.

Hay muchas herramientas que permiten realizar este control de versiones y está sujeto a opinión cuál es mejor. Lo que no está sujeto a opinión es cuál se está utilizando más, lo cual se puede ver en gráficas como la figura \cref{fig:bd:source-control}, que nos facilita \citet{BDSRCMP}. Aparentemente el 70\% de los repositorios públicos está controlado mediante git. 

Otra ventaja de git es que es considerada la herramienta de alto rendimiento. Esto se puede ver también en la figura \cref{fig:g2:source-control}. El origen de esta figura, \citet{G2SRCMP}, nos permite además ir probando distintos parámetros, como el tamaño de la empresa. Y nos muestra que, en cualquier caso, git es la herramienta a elegir para el alto rendimiento.

Así que, con respecto al control de versiones, no hay duda de que git es la herramienta a elegir. Nos permite abarcar la mayoría de desarrollos públicos y es perfecto para el desarrollo de alto rendimiento que necesita un proyecto de calidad. Sin embargo, las decisiones no han acabado aquí. El control de versiones suele subirse a un servidor de repositorios -especialmente si pretende ser código libre-. Este proyecto va a ser código libre, de modo que se puedan proponer cambios y ramificar según distintas necesidades. Para esto, \citet{GHVSGL} explica de forma muy completa que, para proyectos de código abierto, Github es el que tiene mayor comunidad, así que va a tener com más facilidad una repercursión real.

Por tanto, han sido seleccionados git y Github para el control de versiones.

\begin{figure}
	\centering
	\includegraphics[width=\textwidth]{source-control.png}
	\caption{2019 - Compare Repositories}
	\label{fig:bd:source-control}
\end{figure}

\begin{figure}
	\centering
	\includegraphics[width=\textwidth]{source-control-high-perf.png}
	\caption{2019 - Best Version Control Systems}
	\label{fig:g2:source-control}
\end{figure}

\section{Gestión de paquetes}

La gestión de paquetes consiste en el conjunto de herramientas que permiten descargar y mantener las dependencias que el proyecto va a utilizar. Como se ha decidido utilizar React + Express, el entorno de desarrollo es todo Javascript y, por tanto, en servidor se va a utilizar NodeJS, que tiene su propio gestor de paquetes: NPM.

Aunque NPM es una herramienta muy útil en su campo, no es la única que se ha creado para este propósito. Y hoy por hoy, la herramienta competidora por excelencia es Yarn. Como bien explica \citet{NPMVYRN}, existe una diferencia de velocidad a favor de Yarn (ver figura \cref{fig:package-manager:middle-size}). Sin embargo, \citet{NPMVYRN} también nos dice que esa diferencia de velocidad no es grande, así que no debería ser el motivo para elegir un gestor u otro. La funcionalidad de ambos gestores, pese a ser similar, tiene sus diferencias. Además, Yarn sacrifica espacio en disco para ganar ese extra de velocidad que NPM no tiene.

Como en este punto la decisión es tremendamente subjetiva y el coste de implementar ambos es pequeño, para este proyecto me voy a tomar la molestia de contemplar ambas opciones a la vez. Esto quiere decir que, en la configuración inicial del entorno, se presupondrá NPM (que es el gestor por defecto de Node), pero se permitirá al usuario cambiar a Yarn de forma sencilla y rápida.

\begin{figure}
	\centering
	\includegraphics[width=\textwidth]{yarn-vs-npm-middle-sized-apps.jpg}
	\caption{NPM vs Yarn. Speed comparison in a middle-sized app}
	\label{fig:package-manager:middle-size}
\end{figure}


\blinddocument

\chapter{Conclusion and Outlook}
\label{chap:zusfas}

\section*{Outlook}

\printbibliography

All links were last followed on May 19, 2020.

\appendix
\input{latexhints-english}

\pagestyle{empty}
\renewcommand*{\chapterpagestyle}{empty}
\Versicherung
\end{document}
