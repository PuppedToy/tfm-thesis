% !TeX spellcheck = en-US
% !TeX encoding = utf8
% !TeX program = lualatex
% !BIB program = biber
% -*- coding:utf-8 mod:LaTeX -*-


% vv  scroll down to line 200 for content  vv


\let\ifdeutsch\iffalse
\let\ifenglisch\iftrue
\input{pre-documentclass}
\documentclass[
  % fontsize=11pt is the standard
  a4paper,  % Standard format - only KOMAScript uses paper=a4 - https://tex.stackexchange.com/a/61044/9075
  twoside,  % we are optimizing for both screen and two-side printing. So the page numbers will jump, but the content is configured to stay in the middle (by using the geometry package)
  bibliography=totoc,
  %               idxtotoc,   %Index ins Inhaltsverzeichnis
  %               liststotoc, %List of X ins Inhaltsverzeichnis, mit liststotocnumbered werden die Abbildungsverzeichnisse nummeriert
  headsepline,
  cleardoublepage=empty,
  parskip=half,
  %               draft    % um zu sehen, wo noch nachgebessert werden muss - wichtig, da Bindungskorrektur mit drin
  draft=false
]{scrbook}
\input{config}


\usepackage[
  title={Marco de trabajo para la definición de tecnologías aplicables a un proyecto software},
  author={Alejandro Alcázar},
  type={Trabajo de Fin de Máster},
  institute=Universidad Politécnica de Madrid, % or other institute names - or just a plain string using {Demo\\Demo...}
  course={Máster Universitario en Ingeniería Informática},
  examiner={Óscar Dieste},
  supervisor={Óscar Dieste},
  startdate={January 27, 2020},
  enddate={TBD}
]{scientific-thesis-cover}

\newacronym{ssr}{SSR}{Server-Side Rendering}
\newacronym{cms}{CMS}{Content Management System}
\newacronym{seo}{SEO}{Search Engine Optimization}
\newacronym{e2e}{e2e}{End To End}
\newacronym{orm}{ORM}{Object-Relational Mapping}
\newacronym{sql}{SQL}{Structured Query Language}
\newacronym{dod}{DOD}{Document-Oriented Database}
\newacronym{bson}{BSON}{Binary JSON}
\newacronym{json}{JSON}{JavaScript Object Notation}
\newacronym{ddns}{DDNS}{Dynamic Domain Name System}
\newacronym{api}{API}{Application Program Interface}


\makeindex

\begin{document}

%tex4ht-Konvertierung verschönern
\iftex4ht
  % tell tex4ht to create picures also for formulas starting with '$'
  % WARNING: a tex4ht run now takes forever!
  \Configure{$}{\PicMath}{\EndPicMath}{}
  %$ % <- syntax highlighting fix for emacs
  \Css{body {text-align:justify;}}

  %conversion of .pdf to .png
  \Configure{graphics*}
  {pdf}
  {\Needs{"convert \csname Gin@base\endcsname.pdf
      \csname Gin@base\endcsname.png"}%
    \Picture[pict]{\csname Gin@base\endcsname.png}%
  }
\fi

%\VerbatimFootnotes %verbatim text in Fußnoten erlauben. Geht normalerweise nicht.

\input{commands}
\pagenumbering{arabic}
\Titelblatt

%Eigener Seitenstil fuer die Kurzfassung und das Inhaltsverzeichnis
\deftripstyle{preamble}{}{}{}{}{}{\pagemark}
%Doku zu deftripstyle: scrguide.pdf
\pagestyle{preamble}
\renewcommand*{\chapterpagestyle}{preamble}

\section*{Resumen}

<Short summary of the thesis>

\clearpage

\section*{Summary}

<Translated summry of the thesis>

\cleardoublepage


% BEGIN: Verzeichnisse

\iftex4ht
\else
  \microtypesetup{protrusion=false}
\fi

%%%
% Literaturverzeichnis ins TOC mit aufnehmen, aber nur wenn nichts anderes mehr hilft!
% \addcontentsline{toc}{chapter}{Literaturverzeichnis}
%
% oder zB
%\addcontentsline{toc}{section}{Abkürzungsverzeichnis}
%
%%%

%Produce table of contents
%
%In case you have trouble with headings reaching into the page numbers, enable the following three lines.
%Hint by http://golatex.de/inhaltsverzeichnis-schreibt-ueber-rand-t3106.html
%
%\makeatletter
%\renewcommand{\@pnumwidth}{2em}
%\makeatother
%
\tableofcontents

% Bei einem ungünstigen Seitenumbruch im Inhaltsverzeichnis, kann dieser mit
% \addtocontents{toc}{\protect\newpage}
% an der passenden Stelle im Fließtext erzwungen werden.

\listoffigures
% \listoftables

%Wird nur bei Verwendung von der lstlisting-Umgebung mit dem "caption"-Parameter benoetigt
%\lstlistoflistings
%ansonsten:
% \ifdeutsch
%   \listof{Listing}{Verzeichnis der Listings}
% \else
%   \listof{Listing}{List of Listings}
% \fi

%mittels \newfloat wurde die Algorithmus-Gleitumgebung definiert.
%Mit folgendem Befehl werden alle floats dieses Typs ausgegeben
% \ifdeutsch
%   \listof{Algorithmus}{Verzeichnis der Algorithmen}
% \else
%   \listof{Algorithmus}{List of Algorithms}
% \fi
%\listofalgorithms %Ist nur für Algorithmen, die mittels \begin{algorithm} umschlossen werden, nötig

% Abkürzungsverzeichnis
\printnoidxglossaries

\iftex4ht
\else
  %Optischen Randausgleich und Grauwertkorrektur wieder aktivieren
  \microtypesetup{protrusion=true}
\fi

% END: Verzeichnisse


% Headline and footline
\renewcommand*{\chapterpagestyle}{scrplain}
\pagestyle{scrheadings}
\pagestyle{scrheadings}
\ihead[]{}
\chead[]{}
\ohead[]{\headmark}
\cfoot[]{}
\ofoot[\usekomafont{pagenumber}\thepage]{\usekomafont{pagenumber}\thepage}
\ifoot[]{}


%% vv  scroll down for content  vv %%































%%%%%%%%%%%%%%%%%%%%%%%%%%%%%%%%%%%%%%%%%%%%%%%%%%%%%%%%%%%%%%%%%%%%%%%%%%%%%%
%
% Main content starts here
%
%%%%%%%%%%%%%%%%%%%%%%%%%%%%%%%%%%%%%%%%%%%%%%%%%%%%%%%%%%%%%%%%%%%%%%%%%%%%%%


\chapter{Introducción}
\chapter{Introduction}

% \chapter{Chapter Two}
% \label{chap:k2}

% LaTeX hints are provided in \cref{chap:latexhints}.

\chapter{Estudio de mercado de las tecnologías disponibles}
\label{chap:techMarketResearch}

\section{Introducción}

El sistema que se ha planteado ha sido diseñado mediante la siguiente lista de componentes:
\begin{enumerate}
  \item Tecnologías de lado de cliente
  \item Tecnologías de lado de servidor
  \item Control de versiones
  \item Gestión de paquetes
  \item Pruebas unitarias
  \item Pruebas de integración
  \item Pruebas funcionales
  \item Gestor de Base de Datos
  \item Capa de datos
  \item Integración contínua
\end{enumerate}

La figura \cref{fig:architecture} muestra dónde está cada componente y cómo se relaciona con los demás. Por un lado, en la máquina de desarrollo, se podrían encontrar las tecnologías de lado de cliente y servidor (Aplicación y gls{api}), las tecnologías de base de datos y capa de datos y todas las tecnologías de pruebas. La única que quedaría fuera de la máquina de desarrollo es la integración contínua. Además, el entorno de desarrollo cuenta con tecnología de linting, que permite restringir las subidas de código siempre que este no cumpla normativas de limpieza y coherencia.

\begin{figure}
  \centering
  \includegraphics[width=\textwidth]{architecture.png}
  \caption{React Rocket Generator - Architecture}
  \label{fig:architecture}
\end{figure}

Una vez el código cumple los estándares propuestos por el linter y se sube a un repositorio mediante herramientas de control de versiones, se inicia un procedimiento automático descrito en el capítulo de \nameref{section:ci-cd-flow}. En este procedimiento, se utiliza una entidad llamada servidor de integración contínua, que es la responsable de que el flujo se cumpla. En esta entidad vuelven a estar visibles todas tecnologías de pruebas. La máquina de desarrollo tiene acceso a las tecnologías de pruebas como herramienta de desarrollo. Sin embargo, la máquina de integración contínua tiene acceso a estas tecnologías como parte del flujo de integración contínua, y si estás pruebas no son satisfechas, el código no se puede publicar ni en la rama principal ni en el servidor de la aplicación.

Por último, se encuentra el propio servidor de la aplicación, que contiene una versión comprimida del código, junto con la base de datos y la gls{api}. Este servidor es una entidad más compleja, dado que pueden ser varios servidores con un balanceador de carga. Es por eso que ese servidor no entra dentro del marco de desarrollo que se propone y se ha utilizado un icono distinto para representar toda esta complejidad. Cabe destacar que el servidor de integracón contínua y el de la aplicación pueden ser la misma máquina. Sin embargo, este no es el caso habitual y por eso se han marcado como entidades distintas.

Todo el marco está diseñado para que todo lo que hay en la máquina local sea automáticamente configurado mediante una serie de preguntas sencillas. Sin embargo, todo lo que corresponde a las otras máquinas pueden requerir configuración manual. El marco dispone de guías paso por paso para poder configurar todas las conexiones e integraciones descritas en la figura, así como los ficheros de configuración que alimentan el servidor de integración contínua. Todo esto queda explicado más adelante en detalle.

Cabe destacar que este trabajo parte de la premisa de que el marco es solo una propuesta inicial y que, tal y como se explica en el capítulo \nameref{chap:further-steps}, es una propuesta preparada para su expansión.


\section{Entornos de trabajo del lado de cliente}

En esta sección se van a evaluar los entornos de desarrollo más populares en el lado del cliente: React, Svelte, VueJS, Preact y Angular. Como preámbulo, se muestra en la figura \cref{fig:stjs2019:frontend} la satisfacción de los usuarios con respecto a cada entorno. Esto es relevante porque la satisfacción de los usuarios depende de factores como la eficiencia de código y la curva de aprendizaje. Como queremos que nuestro marco de trabajo esté destinado a aplicaciones pequeñas que tienen potencial para crecer, es importante conocer cuál es la opinión media de los usuarios de 2019. Además queremos que el entorno sea accesible al mayor número de usuarios, por lo que la popularidad es el factor más relevante en este estudio.

\begin{figure}
	\centering
	\includegraphics[width=\textwidth]{front_end_frameworks_experience_ranking.png}
	\caption{2019 - Opinión popular de los entornos de trabajo front-end}
	\label{fig:stjs2019:frontend}
\end{figure}

Parece que los dos más importantes en este sentido son React y VueJS (Svelte lo voy a descartar por el poco tiempo en el mercado). Según la comparación de rendimiento que presenta \citet{RWC2019} en su artículo, Vue parece líder en aplicaciones ligeras y rápidas. Por otro lado, en la comparativa que desarrolla \citet{TJSF2019}, se puede apreciar que React es utilizado por un 64.8\% de los desarrolladores web contra los 28.8\% que están con Vue. Además, React cuenta con más soporte y documentación que respaldan sus años de presencia en el mercado, mientras que la para floja de Vue hoy por hoy es su soporte.

Hay otro análisis a tener en cuenta para realizar esta elección, y es la facilidad para depurar el código. La limpieza, la cantidad de líneas y las herramientas de pruebas disponibles cumplemn un papel clave en este trabajo. Para valorarlo, he tenido en cuenta un artículo de \citet{RVVCTOG} que hace la comparativa desde el punto de vista de la calidad de código.

\begin{enumerate}
	\item Tipado: Sabemos que Javascript no es un lenguaje tipado. Sin embargo, cuando hablamos de pruebas, es importante comprobar el tipo de los datos que fluyen a través de la aplicación. Tanto React como Vue permiten comprobaciones de tipos mediante Javascript tipado: TypeScript es una solución global al problema. Sin embargo, es más sencillo usar TypeScript en React. Además, React dispone de una herramienta oficial, Flow, que permite hacer estas comprobaciones de forma todavía más fácil.
	\item Modularidad: Ambos entornos están basados en modularidad de código. Todo depende de lo bien diseñada que esté la aplicación.
	\item Curva de aprendizaje: Vue es líder en este aspecto. Uno de los problemas de React es que su curva de aprendizaje es menos suave que el de otros entornos. Vue está hecho para ser dominado en poco tiempo.
	\item Pruebas y depuración: React es líder en este aspecto. Hay varias herramientas de pruebas, siendo Jest la recomendada por Facebook. Además, React cuenta con una extensión Chrome para depurar sus componentes de forma rápida y sencilla. Aun así, Vue también tiene herramientas de pruebas.
	\item \gls{ssr}: Es importante mejorar la velocidad de descarga y potenciar el \gls{seo}. Para eso, se suele utilizar la técnica de \gls{ssr}. Se puede leer más sobre este tema en el artículo de \citet{SSREXP}. Para este aspecto, ambos entornos tienen sus herramientas. React tiene ahora NextJS y para Vue existe Nuxt.js.
	\item Mantenibilidad del código: Este es uno de los aspectos más importantes, dado que la idea es que el entorno sirva para proyectos pequeños con potencial de crecimiento. El entorno pretende hacer que un proyecto pequeño sea mantenible de forma sencilla, así que este punto es clave. Para este apartado, \citet{RVVMNTB} explica en su artículo que, en sus años de experiencia, React es más mantenible y que leer código de React que han escrito otros es más sencillo. Esto no deja de ser un punto opinable, pero dentro de lo opinable que es, la opinión más extendida aboga para React en aplicaciones más grandes.
\end{enumerate}

Así que, pese a que Vue ha estado creciendo y parece el líder en rendimiento, React parece una elección muy sólida y equilibrada que va a llegar a más desarrolladores y realizar aplicaciones más mantenibles durante el 2020. Por tanto, he decidido realizar el marco de trabajo en React. Si el framework tiene éxito, se planteará extenderlo a otros entornos de trabajo.

Esto tiene repercusiones sobre el esto de toma de decisiones. La figura \cref{fig:react-learnthrough} intenta ilustrar la cantidad de tecnologías que un desarrollador de React debería conocer para dominar todo su potencial. Aun así, esta figura no es más que un mapa conceptual, dado que durante el resto de capítulos se ha ido haciendo una investigación individual de cada tecnología. Aun así, el resultado de las tecnologías en el lado del cliente ha sido bastante parecida a lo que se puede observar en la imagen, así que la incluyo de referencia.

\begin{figure}
  \centering
  \includegraphics[width=\textwidth]{react_learnthrough.png}
  \caption{2020 - El camino del desarrollador de React en 2020}
  \label{fig:react-learnthrough}
\end{figure}



\section{Entornos de trabajo del lado de servidor}

En esta sección se van a evaluar los entornos de desarrollo más populares en el lado del servidor teniendo en cuenta que se va a utilizar React. Si siguiésemos el mismo criterio que con el apartado del lado de cliente, encontraríamos que numerosas fuentes -por ejemplo, \cite{BKETPF1} y \cite{BKETPF2}- están de acuerdo en que PHP es la tecnología más utilizada en el lado de servidor con diferencia. Se puede apreciar en la figura \cref{fig:similartech:backend} sacada de \cite{BKETPF2} en Febrero de 2020. Sin embargo, en este caso particular y por la tecnología elegida en el lado del cliente, se van a valorar positivamente tecnologías que permitan desarrollar al mismo tiempo el cliente y el servidor. PHP es muy utilizado por varias razones:

\begin{enumeration}
	\item Es el más antiguo de los lenguajes de lado de servidor.
	\item Tiene la mayor comunidad que puede tener un lenguaje de servidor y, por tanto, una extensa documentación.
	\item Es utilizado por marcos aun más grandes, como Symphony o Wordpress. Estos son CMS y están fuera del alcance de este proyecto.
\end{enumeration}

Sin embargo, PHP está pensado para directamente renderizar HTML, que es exactamente lo que hace React. Pese a que queremos dar soporte a las tecnologías más extendidas, también queremos estar actualizados con pilas tecnológicas actualizadas y comprobadas. Una de estas pilas tecnológicas de las que hablo es MERN (MongoDB, ExpressJS, ReactJS y NodeJS). Estas pilas tecnológicas están recomendadas por los siguientes motivos:

\begin{enumeration}
	\item Solo hay que aprender un lenguaje, Javascript, lo cual reduce la curva de aprendizaje.
	\item La gestión de datos internos de la aplicación es muy similar a la gestión de base de datos porque MongoDB trabaja con objetos Javascript.
	\item Se pueden hacer contenedores Docker que contengan la pila completa de forma muy sencilla, distribuyendo la carga y haciendo la aplicación escalable. [Faltan citas de todo esto].
\end{enumeration}

Para esto se van a evaluar dos opciones: Express y Next.js. Se ha considerado utilizar múltiples entornos de servidor, entre ellos los que podemos encontrar en la figura \cref{fig:stjs2019:backend}. Sin embargo, teniendo decidido el lado del cliente, la mayoría de opciones no merecen la pena. React trabaja de forma estupenda con NodeJS y lo más habitual es que vayan de la mano. 

\begin{figure}
	\centering
	\includegraphics[width=\textwidth]{similartech_backend_trend_top_10K.png}
	\caption{Leading Framework technologies share on the web - Top 10K Sites}
	\label{fig:similartech:backend}
\end{figure}

\begin{figure}
	\centering
	\includegraphics[width=\textwidth]{back_end_frameworks_experience_ranking.png}
	\caption{2019 - Opinión popular de los entornos de trabajo back-end}
	\label{fig:stjs2019:backend}
\end{figure}

\section{Control de versiones}

El control de versiones consiste en tener todas las versiones de código que se han generado etiquetadas, de forma que se pueda acceder a cualquier versión generada en cualquier momento. Esta es una base imprescindible para un proyecto de calidad, ya que permite trazar errores con mucha facilidad y es una salvaguarda ante fallos. Además, el control de versiones permite tener varias ramas de trabajo, separando el código de la rama de producción, la rama de desarrollo y las distintas ramas de los equipos y características que se están trabjando.

Hay muchas herramientas que permiten realizar este control de versiones y está sujeto a opinión cuál es mejor. Lo que no está sujeto a opinión es cuál se está utilizando más, lo cual se puede ver en gráficas como la figura \cref{fig:bd:source-control}, que nos facilita \citet{BDSRCMP}. Aparentemente el 70\% de los repositorios públicos está controlado mediante git. 

Otra ventaja de git es que es considerada la herramienta de alto rendimiento. Esto se puede ver también en la figura \cref{fig:g2:source-control}. El origen de esta figura, \citet{G2SRCMP}, nos permite además ir probando distintos parámetros, como el tamaño de la empresa. Y nos muestra que, en cualquier caso, git es la herramienta a elegir para el alto rendimiento.

Así que, con respecto al control de versiones, no hay duda de que git es la herramienta a elegir. Nos permite abarcar la mayoría de desarrollos públicos y es perfecto para el desarrollo de alto rendimiento que necesita un proyecto de calidad. Sin embargo, las decisiones no han acabado aquí. El control de versiones suele subirse a un servidor de repositorios -especialmente si pretende ser código libre-. Este proyecto va a ser código libre, de modo que se puedan proponer cambios y ramificar según distintas necesidades. Para esto, \citet{GHVSGL} explica de forma muy completa que, para proyectos de código abierto, Github es el que tiene mayor comunidad, así que va a tener com más facilidad una repercursión real.

Por tanto, han sido seleccionados git y Github para el control de versiones.

\begin{figure}
	\centering
	\includegraphics[width=\textwidth]{source-control.png}
	\caption{2019 - Compare Repositories}
	\label{fig:bd:source-control}
\end{figure}

\begin{figure}
	\centering
	\includegraphics[width=\textwidth]{source-control-high-perf.png}
	\caption{2019 - Best Version Control Systems}
	\label{fig:g2:source-control}
\end{figure}

\section{Gestión de paquetes}

La gestión de paquetes consiste en el conjunto de herramientas que permiten descargar y mantener las dependencias que el proyecto va a utilizar. Como se ha decidido utilizar React + Express, el entorno de desarrollo es todo Javascript y, por tanto, en servidor se va a utilizar NodeJS, que tiene su propio gestor de paquetes: NPM.

Aunque NPM es una herramienta muy útil en su campo, no es la única que se ha creado para este propósito. Y hoy por hoy, la herramienta competidora por excelencia es Yarn. Como bien explica \citet{NPMVYRN}, existe una diferencia de velocidad a favor de Yarn (ver figura \cref{fig:package-manager:middle-size}). Sin embargo, \citet{NPMVYRN} también nos dice que esa diferencia de velocidad no es grande, así que no debería ser el motivo para elegir un gestor u otro. La funcionalidad de ambos gestores, pese a ser similar, tiene sus diferencias. Además, Yarn sacrifica espacio en disco para ganar ese extra de velocidad que NPM no tiene.

Como en este punto la decisión es tremendamente subjetiva y el coste de implementar ambos es pequeño, para este proyecto me voy a tomar la molestia de contemplar ambas opciones a la vez. Esto quiere decir que, en la configuración inicial del entorno, se presupondrá NPM (que es el gestor por defecto de Node), pero se permitirá al usuario cambiar a Yarn de forma sencilla y rápida.

\begin{figure}
	\centering
	\includegraphics[width=\textwidth]{yarn-vs-npm-middle-sized-apps.jpg}
	\caption{NPM vs Yarn. Speed comparison in a middle-sized app}
	\label{fig:package-manager:middle-size}
\end{figure}

\label{section:packet-manager}

\section{Pruebas unitarias}
\label{section:unit-testing}

Las pruebas son una pieza esencial del código de calidad. Sirven para tener la certeza de que al usuario le llega siempre una versión de la aplicación que funciona como se espera. Hay tres tipos de pruebas: las unitarias, las de integración y las funcionales.

En concreto, las pruebas unitarias se encargan de comprobar que un componente de la aplicación funciona como se espera, independientemente del resto de componentes. Si se desea conocer más sobre las pruebas unitarias, recomiendo conocer la historia de \citet{WHYUNT}. 

En cualquier caso, cualquier aplicación con potencial de crecimiento necesita pruebas automáticas. En el mercado hay una infinidad de herramientas que nos resuelven este problema. Aunque en el caso de React, solo hay unas pocas que estén recomendadas oficialmente por el equipo de React. Como se especifica en la documentación (\cite{RETEUT}), se recomienda utilizar Jest junto con una de las siguientes librerías: React Testing Library o Enzyme. Jest parece la opción clara, dado que es la que utiliza Facebook (creadores de React), es la que más se utiliza en el mercado y es la que más satisfacción genera, como se puede comprobar en la figura \cref{fig:stjs2019:unit-testing}.

El dilema surge entre Enzyme y React Testing Library. ¿Qué diferencia hay entre ellos? Como primera respuesta, recomiendo leer la que ha dado \citet{EVRUVRL} en Stack Overflow. Básicamente, Enzyme nos permite acceder a los métodos de nuestros componentes y React Testing Library siempre simula el componente completo. Además, Enzyme permite algo llamado Shallow Render. Esto significa dibujar únicamente el componente que se desea probar, simulando sus hijos. Enzyme es más fácil de utilizar, pero más peligroso y, a la larga, menos mantenible. El hecho de que React Testing Library nos obligue a utilizar los componentes tal y como se renderizan en un ambiente de producción implica tener más seguridad en nuestro código, tal y como explica \citet{NVSHRD} en su artículo y, posteriormente, revisa \citet{RVRT19} en 2019. La filosofía es sencilla: es mejor hacer pruebas que den la confianza de que el código funciona y no emularlo.

Por estos motivos, las pruebas unitarias van a ser mediante Jest y React Testing Library, que son las herramientas recomendadas oficialmente. Durante el transcurso de este trabajo, se valorará utilizar Enzyme haciendo comprobaciones sobre la curva de aprendizaje y la velocidad en proyectos más pequeños, pudiendo ser implementado como opción. Sin embargo, como este entorno pretende ser útil en proyectos con potencial de crecimiento, se preferirá el uso de React Testing Library por su mejor mantenibilidad.

\begin{figure}
	\centering
	\includegraphics[width=\textwidth]{testing_experience_ranking.png}
	\caption{2019 - Opinión popular de las herramientas de pruebas unitarias}
	\label{fig:stjs2019:unit-testing}
\end{figure}

\section{Pruebas de integración}
\label{section:integration-testing}

Las pruebas de integración son aquellas pruebas que permiten comprobar de forma automática que un componente de la aplicación está funcionando bien cuando se utiliza junto con otro componente. Es posible que dos componentes funcionen correctamente cuando están aislados, pero en el momento en el que se unen ocurre funcionalidad inesperada. Las pruebas de integración permiten detectar este fenómeno para poder alertar al desarrollador. Si se desea conocer más sobre las pruebas de integración, recomiendo visitar \citet{DZITWH}.

Durante las pruebas unitarias, hemos hablado de dos librerías de pruebas en React: Enzyme y React Testing Library. De hecho, gran parte de la discusión se ha centrado en un tema que, pese a que no tocaba en ese momento, era indivisible del resto de la comparación. El motivo por el que se ha elegido React Testing Library sobre Enzyme es porque es más preciso en las pruebas de integración. De hecho, Enzyme no realiza pruebas de integración dado que simula la generación de los hijos de un componente.

Así que en React las pruebas de integración y las de unidad están íntimamente relacionadas. Las unitarias se encargarían de comprobar la funcionalidad dentro de un componente React y las de integración, ver cómo se comportan los componentes hijos dentro de distintos componentes padre. En cualquier caso, tal y como nos dice \citet{RECTIT}, las librerías de pruebas unitarias y de integración son las mismas: Jest y React Testing Library.

\section{Pruebas funcionales}
\label{section:e2e-testing}

Las pruebas funcionales o \gls{e2e}, son aquellas pruebas que son totalmente agnósticas de la tecnología utilizada para desarrollar la aplicación. Se encargan de interactuar con ella como si fuesen el usuario final. En el caso de las aplicaciones web, se suelen utilizar navegadores de tipo Headless para realizar estas pruebas de forma automática. Estas pruebas son totalmente independientes de React y NodeJS, aunque lo ideal es desarrollarlas en el mismo lenguaje que el resto de pruebas que se realizan en la aplicación.

Para decidir qué herramienta de pruebas \gls{e2e} se va a utilizar, se han considerado las citadas en el artículo \cite{MPSFT19}. Dentro de estas, se han descartado todas ellas que no tienen compatibilidad con NodeJS para el desarrollo de los casos de prueba o aquellas que no tiene una versión gratuita. Así que se han tenido en cuenta únicamente Selenium y Crypress.

Tal y como explica \citet{CYPVSEL}, Selenium es una herramienta madura y estable, con mucha funcionalidad y que tiene configuración algo más pesada. Sin embargo, Selenium permite grabar las acciones que se realizan sobre el navegador y generar los casos de prueba automáticamente. Y esto es exactamente lo que interesa al marco de trabajo que se propone: ser de configuración rápida y fácil uso para una aplicación pequeña. De la configuración se encargará el propio marco, así que la capacidad de grabar el navegador es una gran ventaja con respecto a Cypress. Además, Selenium está preparado para trabajar con Jest.

Por otro lado, Cypress está específicamente diseñado para entornos de NodeJS y está creciendo a gran velocidad. Como sugiere \citet{CYPVSEL}, Cypress debería considerarse como una alternativa a futuro, que según vaya creciendo se puede ir incorporando a este proyecto. Hoy por hoy, Selenium es líder en herramientas de pruebas \gls{e2e} y, con todo el apoyo que da a entornos de Javascript y Node, no hay ninguna otra que sea apropiada para la tarea.


\section{Gestor de Base de Datos}

Esta decisión es probablemente la más difícil de todas las que se han tomado en este proyecto. Por un lado, la gestión de bases de datos depende mucho de los conocimientos técnicos del desarrollador. Por otro, la base de datos puede acelerar mucho el tiempo de desarrollo en función de la afinidad que haya entre los datos y el gestor elegido. Además, hay gestores que son más escalables o tienen distintas prestaciones en función del resto de tecnologías elegidas.

Para tomar esta decisión, se han abordado los puntos con sumo cuidado, comenzando por un análisis de popularidad, tal y como se ha hecho en las anteriores secciones. Según el análisis de \citet{MPDBITW} (resumido en la figura \cref{fig:dbms}), los gestores de bases de datos más populares son:
\begin{enumerate}
	\item Oracle
	\item MySQL
	\item Microsoft SQL Server
	\item PostgreSQL
	\item MongoDB
\end{enumerate}
Las cuatro primeras opciones son todo bases de datos relacionales. En quinta posición se encuentra MongoDB, que es la única opción no relacional dentro del top de popularidad. Como uno de los objetivos de este marco es llegar al mayor número de desarrolladores posible, se ha intentado mantener el marco dentro de estos gestores de bases de datos. Además, otro factor a tener en consideración es que no se va a incluir ninguna herramienta que no sea gratuita, así que Oracle y Microsoft SQL Server quedan descartadas. Esto reduce las posibilidades a tres gestores de base de datos: MySQL, PostgreSQL y MongoDB.

MySQL y PostgreSQL son, como comenté, bases de datos relacionales que utilizan un lenguaje común: \gls{sql}. Además, hay otros muchos gestores que utilizan este lenguaje de consultas y cada usuario puede preferir uno u otro. Hay una discusión abierta sobre cuál es mejor y hay defensores y detractores de cada uno. Se puede leer un poco más de este tema en la reflexión de \citet{MYSVPOS}.

MongoDB es un gestor no relacional (en concreto, se le conoce como lenguaje No \gls{sql}). Esto quiere decir que no se basa en un modelo estricto. Específicamente, MongoDB es un \gls{dod}. Se puede leer más sobre este tipo de gestores en la propia documentación de MongoDB (\cite{DOCORDB}). Lo que hace a MongoDB tan buen gestor de base de datos para entornos NodeJS es que esos documentos que gestiona se guardan en formato \gls{bson}, que es una forma de guardar en disco de forma eficiente documentos \gls{json}. La forma de comunicarse con este gestor es enviando precisamente objetos con exactamente la misma sintaxis que los objetos en JavaScript. Así que es un lenguaje estupendo para trabajar desde entornos con JavaScript.

Por otro lado, hay que tener en cuenta los \gls{orm}, que son herramientas que hacen de capa intermedia entre la aplicación y la base de datos. Permiten modelizar los datos como si fuesen objetos y se encarga de mantener la coherencia entre la base de datos y la estructura proporcionada. Realizar operaciones básicas es muy directo utilizando un \gls{orm}, pero aumentan la dificultad y reducen la eficiencia de operaciones complejas. Por tanto, utilizar o no un \gls{orm} depende mucho de las necesidades de la aplicación. Se puede leer más sobre los criterios de utilización de un \gls{orm} en los artículos de \citet{ORMYON1} o \citet{ORMYON2}. Sin embargo, los \gls{orm} permiten utilizar la misma sintaxis para comunicarse con bases de datos diferentes, lo cual es algo que beneficia mucho a un marco de lanzamiento rápido como se está desarrollando. Si buscásemos a un \gls{orm} adecuado, TypeORM podría ser un ORM adecuado porque tiene soporte a MySQL, PostgreSQL y MongoDB a la vez.

Como se puede comprobar, cada tecnología en este apartado tiene sus ventajas e inconvenientes. Modelos relacionales permiten una estructura consistente en base de datos, mientras que MongoDB permite introducir en la base de datos objetos que tiene guardados en su memoria interna sin ninguna transformación. Por otro lado, el modelo puede requerir otros tipos de gestores. Los datos podrían requerir otros modelos (ver artículo de \citet{DBMSTYP}) o los desarrolladores podrían estar más familiarizados con algún otro en particular. Es importante recalcar que no debería ser tarea del marco elegir la tecnología que debería utilizar el usuario para la permanencia de información. Sin embargo, sí que es tarea del marco reducir al mínimo el tiempo que emplea un usuario en la configuración inicial del entorno. El marco está principalmente orientado a la calidad de código. Esto puede conseguirse independientemente del gestor de base de datos elegido.

Por todo esto, se ha decidido que el marco va a dar soporte a una configuración rápida para un entorno con MySQL, PostgreSQL, MongoDB o TypeORM. Será una opción, se gestionará localmente en la máquina que está preparada para alojar el proyecto y siempre será un elemento de libre configuración. De este modo, si la aplicación no necesita permanencia de datos o necesita una más especializada, el marco no pondrá límites en este aspecto.

\begin{figure}
	\centering
	\includegraphics[width=\textwidth]{popular_databases.jpg}
	\caption{2019 - Most Popular Databases In The World}
	\label{fig:dbms}
\end{figure}

\label{section:database-manager}

\section{Capa de datos}

La persistencia de datos es muy importante de cara a mantener una aplicación web. Sin embargo, en los marcos de trabajo actuales, es habitual descargar mucha carga de datos en el cliente y que este lo vaya administrando. Esta gestión puede ser muy tediosa si no se utilizan unas tecnologías llamadas capas de datos, que permiten mantener la coherencia de los datos en toda la aplicación. Las herramientas más populares se muestran en la Figura \cref{fig:stjs2019:datalayer}. 

Se puede apreciar que GraphQL es líder en este sector, y no por pocos motivos. Esta herramienta fue creada por Facebook (al igual que React y su principal competidora, Redux) y ha nacido a raíz de algunos problemas que surgían al utilizar React. Sin embargo, no está diseñada exclusivamente para React. GraphQL se puede utilizar junto con cualquier entorno. Dado que esta herramienta es la más popular en el mercado, está diseñada por los creadores de React y resuelve los problemas que explica \citet{RDXVGQL} en su artículo, es la herramienta elegida para el marco que se ha desarrollado.

Sin embargo, no siempre se necesita capa de datos en el cliente y, por tanto, la posibilidad de utilizar GraphQL será una configuración opcional. En la configuración inicial, al desarrollador se le preguntará si desea capa de datos (GraphQL) en su entorno y este podrá sencillamente contestar si lo desea o prefiere desarrollar su producto controlando la capa de datos de forma personalizada, ya sea mediante otra herramienta o cuidando el estado y las propiedades de sus componentes de React.

\begin{figure}
	\centering
	\includegraphics[width=\textwidth]{data_layer_popularity.png}
	\caption{2019 - Opinión popular de las herramientas de gestión de la capa de datos}
	\label{fig:stjs2019:datalayer}
\end{figure}

\label{section:data-layer}

\section{Flujo de despliegue - Integración continua}
\label{section:ci-cd-flow}

La integración contínua es una disciplina que permite automatizar el flujo de código desde el momento en que un desarrollador lo genera hasta que llega a la rama principal de código del repositorio, asegurando por el camino que es código de calidad mediante pruebas automáticas mencionadas en las secciones \nameref{section:unit-testing}, \nameref{section:integration-testing} y \nameref{section:e2e-testing}. Además, uniéndolo con el despliegue contínuo, se puede lograr que haciendo solo un click el código sea verificado, enviado a un servidor y expuesto al público. \citet{CVTRVJK} explica este procues en su artículo, mostrando además dos gráficos muy representativos de cómo funcionan la integración contínua (\cref{fig:ci}) y el despliegue contínuo (\cref{fig:cd}).

En la figura \cref{fig:ci} se puede observar cómo múltiples desarrolladores pueden colaborar simultáneamente enviando sus aportaciones a un servidor de control de versiones como GitHub. Mediante un simple evento PUSH en una rama de desarrollo, los cambios irán al repositorio, que emitirá la nueva versión a un servidor de integración contínua. El servidor verificará que el código cubre todos los casos de prueba y, en caso de fracasar, notificará a los desarrolladores involucrados en la nueva versión, que podrán arreglar los fallos y volver a intentar una subida. En caso de éxito, podrá notificar a los desarrolladores también. Sin embargo, lo más habitual es que un éxito desencadene el siguiente proceso, el despliegue contínuo, que se encargará de subir la nueva versión a un servidor público de forma inmediata. Cabe remarcar que este servidor público puede estar duplicado (uno para desarrollo y otro para producción), generando dos procesos de integración contínua que pueden estar tanto encadenados como construidos de forma independiente.

En la figura \cref{fig:cd} se muestra el proceso completo, incluyendo la integración contínua y el despliegue contínuo. En este flujo pueden intervenir revisiones manuales de código y pruebas por parte de los usuarios. Este proceso asume que la aplicación es desarrollada en un entorno profesional, pero el marco que se plantea desarrollar no está pensado en un ámbito de empresa. El marco pretende abarcar proyectos pequeños (de pocos desarrolladores) con potencial de crecer. Por tanto, este flujo debe tenerse en consideración, pero no debe ser la principal preocupación del marco.

La idea es que el marco proporcione de forma sencilla tanto un entorno de integración contínua como un entorno de despliegue contínuo. Es objeto de estudio de este marco analizar las distintas herramientas que permiten hacer estas tareas y pensar en cómo podría un proyecto escalar a ámbito profesional sin modificar las herramientas que se utilizan en el proceso. Para poder realizar estas tareas, se han analizado las dos herramientas gratuitas que propone \citet{CVTRVJK}: Jenkins y CircleCI. TravisCI se descarta por su carácter de pago, que rompe con la filosofía del marco.

\begin{figure}
	\centering
	\includegraphics[width=\textwidth]{how-ci-works.png}
	\caption{How Continuous Integration Works}
	\label{fig:ci}
\end{figure}

\begin{figure}
	\centering
	\includegraphics[width=\textwidth]{release-workflow.png}
	\caption{Release Workflow}
	\label{fig:cd}
\end{figure}

\chapter{Desarrollo del marco de trabajo}
\label{chap:development}

\section{Funcionamiento general}
El marco de trabajo ha sido diseñado para ser lo más automático posible. Para realizar todas estas automatizaciones, NodeJS dispone de un gestor de paquetes (que se ha discutido en \nameref{section:packet-manager}). Este gestor se encarga de tener controladas todas las dependencias del proyecto. Para ello, utiliza un fichero mantenido por el equipo de desarrollo llamado package.json. Además, se genera otro fichero package-lock.json a partir del primero, que contiene las versiones exactas del proyecto.

Este fichero package.json tiene un valor muy especial para los desarrolladores de NodeJS porque además contiene otros campos con otras finalidades. Uno de los más importantes es el campo scripts, que permite poner a disposición mandatos de consola que realizan ciertas funcionalidades. El objeto scripts es un conjunto de pares clave-valor, donde la clave es el nombre del mandato y el valor es el comando bash que se va a ejecutar al utilizar ese mandato. Tenga en cuenta que este comando bash permite el uso de binarios que se encuentren dentro de las dependencias del proyecto, por lo que la potencia de estos scripts es infinita.

Con esta base, se ha decidido incluir en el proyecto uno de estos mandatos: el mandato setupº, que se encarga de montar el proyecto en función de las preferencias del usuario. Para ello, utiliza la librería Inquirer \cite{NPMLINQ}, que permite hacer una serie de preguntas al usuario mediante la consola. Una vez terminadas las preguntas, el mandato setup automáticamente elimina todos aquellos ficheros del repositorio que el usuario no va a utilizar (en función de sus preferencias), enlaza los que sí va a utilizar y deja preparados los scripts que el usuario pueda necesitar, de forma que el usuario pueda desde ese momento ponerse a trabajar. Se puede observar cómo se realizan las preguntas al usuario en las figuras \cref{fig:rocket-generator-1} y \cref{fig:rocket-generator-2}. En la figura \cref{fig:rocket-generator-3} se puede ver el final del script con todas las preguntas realizadas al usuario.

Cabe destacar que el proyecto se ha planteado como un proyecto abierto y la intención es ir incrementando el número de opciones de las que dispone el usuario para que el marco vaya siendo cada vez lo más completo posible. Cómo se puede ampliar el proyecto y cómo se ha planteado la ampliabilidad del proyecto se detalla más adelante en el apartado \nameref{chap:further-steps}.

El marco cuenta con un proyecto modelo para todas las opciones que introduzca el usuario. El proyecto modelo es una simple lista de tareas (ver lista de tareas vacía en la figura \cref{fig:todo-list-1}), en la cual el usuario puede añadir nuevas tareas (\cref{fig:todo-list-2}), marcar y desmarcar las tareas como hechas (\cref{fig:todo-list-3}), modificarlas y borrarlas. Este proyecto modelo ha sido realizado para todas las opciones de las que dispone el marco de trabajo: todos los tipos de base de datos incluídos en \nameref{section:database-manager} y todas las capas de datos incluídas en \nameref{section:data-layer}. Por el momento, las pruebas se han realizado únicamente mediante Jest y React Testing Library (tal y como se comenta en \nameref{section:unit-testing}), pero entra dentro del ámbito del proyecto (tal y como se comenta más adelante en \nameref{chap:further-steps}) dar también soporte a Enzyme. Esto implicará repetir todas las baterías de pruebas para Enzyme.

\begin{figure}
  \centering
  \includegraphics[width=\textwidth]{rocket-generator-1.png}
  \caption{React Rocket Generator - Asking about Package Manager}
  \label{fig:rocket-generator-1}
\end{figure}

\begin{figure}
  \centering
  \includegraphics[width=\textwidth]{rocket-generator-2.png}
  \caption{React Rocket Generator - Asking about the API}
  \label{fig:rocket-generator-2}
\end{figure}

\begin{figure}
  \centering
  \includegraphics[width=\textwidth]{rocket-generator-3.png}
  \caption{React Rocket Generator - All questions together}
  \label{fig:rocket-generator-3}
\end{figure}

\begin{figure}
  \centering
  \includegraphics[width=\textwidth]{todo-list-1.png}
  \caption{React Rocket Todo List - Empty List}
  \label{fig:todo-list-1}
\end{figure}

\begin{figure}
  \centering
  \includegraphics[width=\textwidth]{todo-list-2.png}
  \caption{React Rocket Todo List - Generated 3 items}
  \label{fig:todo-list-2}
\end{figure}

\begin{figure}
  \centering
  \includegraphics[width=\textwidth]{todo-list-3.png}
  \caption{React Rocket Todo List - Item marked as done}
  \label{fig:todo-list-3}
\end{figure}


\section{Detalle de la integración continua}
\label{chap:ci-cd-detail}
Para el desarollo de la integración continua y el despliegue continuo se han utilizado dos herramientas libres: Jenklins y CircleCI. El objetivo era minimizar el esfuerzo de configuración, pero esta tarea se dificulta sabiendo que, tal y como explica el diagrama de la figura \cref{fig:architecture}, puede haber un total de 4 máquinas interviniendo en el despliegue. Una de las cuales es el servidor de git y se asume que se está utilizando Github, así que se descarta. La máquina que se encarga del desarollo continuo es necesaria solo si se utiliza Jenkins o si se utiliza el plan de pago de CircleCI. Como se ha ido comentando a lo largo del trabajo, se descarta cualquier opción de pago, así que se va a asumir que si se utiliza Jenkins, se dispone de una máquina para el servidor de integración continua y si se utiliza CircleCI no.

Tanto jenkins como CircleCI disponen de la posibilidad de configurar el proyecto mediante un fichero de configuración que se incluye dentro del repositorio. Esto implica que el usuario no tendrá que preocuparse por casi nada de la configuración del proyecto. En el caso de Jenkins, este fichero se llama Jenkinsfile y utiliza la sintaxis Groovy y en el caso de CircleCI, este fichero se llama .circleci/config.yml y utiliza la sintaxis YAML. Sin embargo, ambas herramientas carecen de un método para generar automáticamente el proyecto, así que se requiere un mínimo trabajo manual para generar el proyecto y conectarlo al repositorio. En el caso de Jenkins, además, es necesario el plugin de Github y el plugin SSH Publisher para poder realizar una integración completa. Jenkins es ligeramente más complicado de montar porque requiere tener una máquina disponible para hacer de servidor. Sin embargo, su instalación es directa y se ejecuta únicamente mediante el mandato jenkins --daemon.

Respecto a la máquina que va a alojar el código, se espera que posea por lo menos los siguientes paquetes:
\begin{enumerate}
  \item git
  \item node y npm
  \item pm2: esta herramienta permite automatizar el despliegue en un solo fichero de configuración. El marco de trabajo asume que se utiliza pm2 para poder desplegar el código mediante un solo mandato.
  \item ssh: es necesario para poder comunicarse con la máquina.
  \item Visibilidad en la red: esto no es un paquete como tal, pero es un requisito indispensable. En el caso de Jenkins, la máquina objetivo debe estar dentro de la misma red (o ser la misma máquina) que la máquina que aloja Jenkins. En el caso de CircleCI, es imprescindible que la máquina tenga una dirección IP estática o tenga asociado un nombre de dominio a una IP Dinámica mediante \gls{ddns}. Además, deberá tener abiertos el puerto 80 y el 8080.
\end{enumerate}
Tal y como se comenta en \nameref{chap:further-steps}, se pretende dar soporte a Docker. Una vez se logre este hito, será indispensable que la máquina de la aplicación tenga instalado Docker. Para estos requisitos mínimos se ha elaborado un fichero de instalación que habrá que ejecutar una sola vez en la máquina que vaya a alojar el código, asumiendo que la máquina destino tiene acceso a apt. En caso de tener una distribución con otro servidor de paquetes, el usuario deberá buscar por su cuenta la forma de instalar los paquetes mencionados anteriormente. Dada una máquina con las características mencionadas, el fichero generado tanto para Jenkins como para CircleCI será capaz por sí mismo de ordenar a la herramienta en cuestión el despliegue automático en el servidor que vaya a alojar la aplicación.

Para simplificar el despliegue, se asume que la máquina que vaya a alojar la aplicación va a hacerlo utilizando el puerto 80 (o 443 en el caso de HTTPS) y que, la misma máquina va a utilizar el puerto 8080 para alojar la misma aplicación en preproducción. Además, se asume que la rama develop del servidor de git se conectará al servidor de preproducción y que la rama master se conectará al servidor de producción. Todo esto ya está contemplado en los ficheros de configuración tanto de Jenkins como de CircleCI.

Quiero aclarar que soy consciente de que esta es sin duda la parte más engorrosa del proyecto. Por su propia naturaleza, la variabilidad del entorno, la intervención de la red y el uso de herramientas muy gráficas como Jenkins y CircleCI, la cantidad de trabajo manual necesaria para poner en marcha el proyecto es mucho mayor de la que desearía. A lo largo del proyecto me he ido dando cuenta de la gran necesidad que hay de automatizar aun más todo este proceso y, por tanto, me he dado cuenta de que Docker debería haber sido parte del proyecto desde el principio. Sin embargo, mis recursos me han impedido añadirlo en esta primera fase y, por tanto, lo he dejado como máxima prioridad en el apartado de \nameref{chap:further-steps}. Docker no permitirá toda la automatización del sistema, pero sí permitirá una mejor integración con las herramientas mencionadas en este capítulo y reducirán mucho la carga de instalación de paquetes en la máquina que aloja la aplicación.


\chapter{Métricas del marco de trabajo}
\label{chap:metrics}

\section{Introducción}
Una vez ha sido construido el marco de trabajo, se ha realizado el proyecto "TODO List" con él, tal y como se explica en el apartado \nameref{chap:development}. Este proyecto se ha realizado de dos formas distintas: primero se ha realizado sin el marco de trabajo, y luego se ha repetido utilizando el nuevo marco.

El objetivo de este experimento es comprobar cuánto esfuerzo supone realizar un proyecto muy pequeño de forma que el resultado tenga calidad y sea escalable y, de ese modo, comprobar desde qué tipo de proyecto puede empezar a merecer la pena aplicar dicho esfuerzo.

La motivación original de este proyecto era conseguir reducir este esfuerzo inicial que supone realizar este trabajo mantenible y escalable. Lo ideal para este experimento habría sido que hubiese una población más grande y dividirla en tres grupos de desarrollo:

\begin{enumerate}
\item Los que hacen el proyecto sin aportar calidad.
\item Los que hacen el proyecto a partir del marco de trabajo y sin conocerlo previamente.
\item Los que hacen el proyecto con calidad, pero sin utilizar el marco de trabajo.
\end{enumerate}

De este modo, podríamos realizar un verdadero estudio estadísitico sobre lo que aporta de verdad el marco. Sin embargo, dados los recursos de este trabajo (tanto de tiempo como de capacidad de trabajo), tengo que reducir el experimento a un solo desarrollador (yo mismo) y prescindir del tercer grupo de trabajo. Además, una vez hecho el proyecto con marco de trabajo, al realizar el proyecto sin marco de trabajo ya habré pasado por un proceso previo de diseño y habré aprendido de mis errores en el desarrollo anterior. Por todo esto, la valided del experimento debe ser tomada con cuidado.

Aun así, se pueden sacar conclusiones de este experimento. El objetivo es tener una idea general de cuánto supone este esfuerzo, hacer una estimación de costes para una empresa pequeña y, de paso, comprobar si he echado de menos algo del marco de trabajo para poder tenerlo en consideración a futuro. Para poder sacar el máximo partido al experimento, en este apartado he realizado predicción sobre los resultados que esperaba y por qué antes de realizar dicho experimento. Una vez hecho el experimento, he complementado las predicciones con comentarios sobre las sorpresas que he recibido y las conclusiones que he sacado de los números obtenidos.


\section{Predicciones: resultados esperados}
Todas las estimaciones que se van a poder ver a continuación han sido realizadas bajo la asumción de que la persona que realiza las tareas domina las tecnologías que está utilizando. Para este proyecto se han estimado los siguientes tiempos sin utilizar el marco de trabajo ni aportando calidad al producto:

\begin{enumerate}
  \item Servidor: 3 horas
    \begin{enumerate}
      \item Montar el servidor (GraphQL y Express): 1 hora
      \item Añadir tarea: 0.5 hora
      \item Eliminar tarea: 0.5 hora
      \item Editar tarea: 0.5 hora
      \item Marcar tarea: 0.25 horas
      \item Obtener lista de tareas: 0.25 horas
    \end{enumerate}
  \item Cliente: 5.5 horas
    \begin{enumerate}
      \item Montar el cliente (Webpack): 2 horas
      \item Página de lista de tareas: 0.5 horas
      \item Componente de campo de texto: 0.5 horas
      \item Componente de botón de borrar tarea: 0.25 horas
      \item Componente de botón de editar tarea: 0.5 horas
      \item Componente de tarea: 0.25 horas
      \item Componente de caja de tareas: 0.25 horas
      \item Componente de botón de añadir tarea: 0.25 horas
      \item Integración con el servidor: 1 hora
    \end{enumerate}
  \item Subir a producción: 30 minutos
\end{enumerate}

En total, sin utilizar el marco de trabajo, se estima que se va a tardar 9 horas. En principio, en un día largo de trabajo podría abordarse el proyecto TODO List entero. El marco de trabajo se encargará de ahorrarnos las matesde montar el servidor, montar el cliente y la integración con el servidor. A cambio, nos exigirá un mínimo de cobertura en el código y nos facilitará engancharnos con un servidor de integración contínua. Por tanto, se espera que al utilizar el marco de trabajo el coste del proyecto sin preubas se reduzca en 3.5 horas aproximadamente. A cambio, se esperan los siguientes costes en pruebas:

\begin{enumerate}
  \item Pruebas unitarias en servidor: 5 horas
  \item Pruebas unitarias en cliente: 10 horas
  \item Pruebas de integración: 4 horas
  \item Pruebas funcionales: 6 horas
\end{enumerate}

Por tanto, se espera que el proyecto va a costar 32 horas utilizando el marco de trabajo para aportar calidad. Hay que tener en cuenta que, una vez terminado el proyecto con el marco de trabajo, se asume integración contínua y automática con Github, Jenkins, entorno de preproducción y producción. Cuanto más se modifique este proyecto a futuro, más rentable saldrá montar todo este entorno de calidad.

Además de los tiempos, es muy importante medir errores. En un proyecto tan pequeño, se esperan en total 10 errores máximo sin utilizar el marco de trabajo. En el proyecto con marco de trabajo no se espera ningún error.


\section{Resultados obtenidos}
// TODO

\section{Conclusiones}
// TODO

\chapter{Próximos pasos}
\label{chap:further-steps}

\section{Abrir el código}
El alcance del trabajo está limitado a las horas impuestas y, dadas las circunstancias, se ha decidido acotar a unas pocas tecnologías elegidas generalmente mediante el criterio de la popularidad, de forma que este marco pueda ser útil al mayor número posible de desarrolladores. El código ya ha sido escrito con eso en mente y, por tanto, se ha preparado el terreno para dar un paso evidente en el ciclo: publicar el código como abierto.

Esto tiene las siguientes ventajas:
\begin{enumerate}
  \item El código es revisado por numerosos desarrolladores, dando distintos puntos de vista y aportando diversidad.
  \item El código se produce a más velocidad y los errores se resuelven más rápido, sin coste alguno.
  \item El proyecto se puede reutilizar y se pueden generar otros marcos y otras ideas.
  \item El marco se puede ampliar a más tecnologías porque más usuarios lo van a dar forma en función de sus necesidades.
\end{enumerate}

Sin embargo, el código abierto no es un paso pequeño. Tal y como constata el estándar de código abierto \cite{OPENSRC}, tiene una serie de responsabilidades para el dueño originario del código para que pueda funcionar, que en este marco son necesarias. Estas responsabilidades son, en general, buenas prácticas para cualquier código, pero en este marco en particular son unos requisitos indispensables a ser abordados antes de poder publicar el código como abierto:
\begin{enumerate}
  \item Todo el código debe estar revisado y comentado en función de una guía de buenas prácticas. La dificultad de esta tarea se va a reducir porque se utilizó Linter desde el primer momento, pero aun así una revisión de calidad es necesaria.
  \item El marco debe estar documentado en su totalidad. En el estado actual del proyecto, toda la documentación se refiere a como construir un proyecto a partir del marco, pero el marco en sí debe ser documentado mediante el punto de vista del editor del marco. Esto implica documentar las tareas automáticas y una guía de amplicación de las tencologías disponibles. Para hacer estas guías, es razonable realizar una amplicación por mí mismo para comprobar que todo funciona como se había planteado.
  \item Elaborar los documentos requeridos para el código abierto: Licencia, README, Guía de contribución y Código de conducta. El README ya está generado, pero requerirá una revisión y quizá nuevos apartados para la instalación desde un punto de vista de contribución.
\end{enumerate}

En el momento en el que el código se transforme en código abierto, se habrá cumplido el verdadero propósito del proyecto y se considerará iniciada la primera versión oficial del producto. Por tanto, el código ha de estar totalmente completo desde todos los puntos de vista posibles. El estado actual del producto es que es completo y listo para ser utilizado como marco, como semilla de otros proyectos. Pero faltan una serie de pasos para poder ser ampliado libremente y esto requiere planificación y revisión, así que esto será, en definitiva, el próximo paso más inmediato.


\section{Ampliabilidad del código}
Uno de los pasos fundamentales para poder publicar el código abierto es que sea ampliable. Como la idea es que la tarea de ampliar el marco se va a repetir en numerosas ocasiones, se ha planteado la opción de automatizarla en la medida de lo posible. Por tanto, teniendo en cuenta el futuro, he diseñado el proceso de ampliación y las tareas que van a facilitar la tarea. Supongamos que un usuario desea ampliar el código con una tecnología particular -digamos que por ejemplo desea añadir Redis como base de datos- así que se clona el respositorio y se instala las dependencias. Lo ideal sería que ese usuario ejecutase una tarea framework:expand y que, mediante el mismo sistema que utiliza el marco para comenzar un proyecto, le hiciese unas preguntas sobre su ampliación.

Estas preguntas buscarán obtener la siguiente información:
\begin{enumerate}
  \item Conocer el nombre de la tecnología a implementar. En este caso Redis
  \item Conocer el tipo de tecnología a implementar. En este caso Base de datos.
  \item Conocer qué ficheros implicados ha de poseer el marco, que deberá eliminar si el usuario no elige la tecnología.
  \item Conocer qué tarea del package.json deberá aplicar al instalar el marco en caso de necesitarla. En este caso particular, el usuario puede querer producir una instancia de la base de datos en su instalación.
  \item Conocer las opciones con las que se puede utilizar esta tecnología en el marco. Estas opciones serán luego ofrecidas al usuario final durante la fase de preguntas si elige utilizar esta nueva tecnología, Redis, y el resultado de la selección se le pasará a esa tarea de inicialización del package.json para que el usuario desarrollador pueda trabajar con ellas.
\end{enumerate}

Con toda esta información, el marco será capaz de mantener un registro de los ficheros asociados, podrá pasar pruebas automáticas con distintas configuraciones y podrá actualizar una documentación online sobre las tecnologías disponibles. Todas estas tareas quedan a responsabilidad del propio marco y no del usuario.

Como la intención no es restringir, si no ayudar, esta tarea debe ser una mera inicialización de la nueva tecnología, pero todo debe poder ser editado al vuelo. Por tanto, esta tarea no hará más que generar información en una serie de ficheros que podrán ser editados. Estos ficheros serán la clave de la ampliabilidad, dado que permiten documentar de forma automática el marco y, a la vez, automatizar todas las tareas de instalación.

Por otro lado, el usuario quedará responsable de cumplir los requisitos de cobertura para su tecnología en la aplicación de pruebas "Lista de tareas", generar las tareas automáticas de instalación y mantener los ficheros mencionados en el párrafo anterior. Aun así, una vez el código pasa esas pruebas y está listo para ser subido, requerirá una revisión manual por parte de los encargados del mantenimiento del código, que será un equipo compuesto por personas de confianza que hayan contribuido al código. En la imagen inicial, estas ampliaciones solo requerirán de mi revisión manual.

La idea es que, de forma inicial, la ampliación del código sea lo más sencilla y automática posible. Y que, poco a poco, con contribuciones de distintos desarrolladores, el sistema se vaya cerrando más y quede totalmente automatizado. Todavía no se ha diseñado exactamente qué ficheros van a participar y qué estructura van a tener, pero esto será uno de los pasos requeridos antes de poder publicar el código como abierto.


\section{Próximas tecnologías a abordar}
Una vez se haya abierto el código, mi tarea con el marco no ha concluido. Además de las revisiones que tendré que hacer y de la resolución de errores que vayan surgiendo, quedará a mi responsabilidad contribuir a la ampliación del marco por mí mismo. De este modo, podré probar si todo lo que he diseñado para la ampliación funciona bien y tiene sentido. Además, se publicará una serie de tecnologías que han sido contempladas en la fase de investigación pero no han podido ser incluídas en el marco original por falta de tiempo y alcance en el trabajo.

Queda a mi responsabilidad el ir añadiendo poco a poco estas tecnologías a menos que algún contribuyente decida abordar alguna por sí mismo. Esta lista ya ha sido pensada en la fase de investigación, pero quedará expuesta a posibles cambios en función de la demanda en el apartado Issues de Github, que tendrá una etiqueta disponible para pedir ampliaciones de tecnologías. Mi tarea consistirá en ir añadiendo las más populares, independientemente de la lista que voy a añadir a continuación. Sin embargo, por el momento y mientras el marco no tenga suficiente público, se irán abordando en el orden en que considero más necesario a raíz de este trabajo.

Es evidente que la dirección concreta que va a tomar el marco depende en su totalidad de su popularidad. En el desafortunado caso de que este marco no logre darse a conocer o no logre ser de suficiente utilidad, el marco quedará completo cuando las tecnologías de la lista hayan sido añadidas. En caso de que el marco adquiera uso y la demanda aumente, es probable que la lista cambie de forma en función de esta demanda.

Así que, como último apartado de los últimos pasos, será añadir las siguientes tecnologías en el orden propuesto:
\begin{enumerate}
  \item Docker: es imprescindible que el marco de soporte a contenedores Docker y, por tanto, esta será la máxima prioridad.
  \item Enzyme: numerosos desarrolladores están trabajando con Enzyme en lugar de React Testing Library. Me parece requisito indispensable que acabe entrando dentro del marco como máxima prioridad.
  \item Redis: una base de datos libre que se utiliza en numerosos proyectos.
  \item Vue
  \item AngularJS
  \item Go: esta idea todavía no la tengo clara, dado que todo el marco está en NodeJS. Sin embargo, sería adecuado si otros lenguajes pudiesen ser utilizados en el lado de servidor. Go parece ser la alternativa a NodeJS que más está entrando en el mercado. De todos modos, esta opción queda pendiente a ser evaluada cuando llegue el momento.
\end{enumerate}


\chapter{Conclusiones del proyecto}

\printbibliography

Todos los enlaces han sido comprobados el 22 de Febrero de 2020.
% Todos los enlaces han sido comprobados el May 19, 2020.

% \appendix
% \input{latexhints-english}

\pagestyle{empty}
\renewcommand*{\chapterpagestyle}{empty}
% \Versicherung
\end{document}
