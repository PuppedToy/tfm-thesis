Una vez se haya abierto el código, mi tarea con el marco no ha concluido. Además de las revisiones que tendré que hacer y de la resolución de errores que vayan surgiendo, quedará a mi responsabilidad contribuir a la ampliación del marco por mí mismo. De este modo, podré probar si todo lo que he diseñado para la ampliación funciona bien y tiene sentido. Además, se publicará una serie de tecnologías que han sido contempladas en la fase de investigación pero no han podido ser incluídas en el marco original por falta de tiempo y alcance en el trabajo.

Queda a mi responsabilidad el ir añadiendo poco a poco estas tecnologías a menos que algún contribuyente decida abordar alguna por sí mismo. Esta lista ya ha sido pensada en la fase de investigación, pero quedará expuesta a posibles cambios en función de la demanda en el apartado Issues de Github, que tendrá una etiqueta disponible para pedir ampliaciones de tecnologías. Mi tarea consistirá en ir añadiendo las más populares, independientemente de la lista que voy a añadir a continuación. Sin embargo, por el momento y mientras el marco no tenga suficiente público, se irán abordando en el orden en que considero más necesario a raíz de este trabajo.

Es evidente que la dirección concreta que va a tomar el marco depende en su totalidad de su popularidad. En el desafortunado caso de que este marco no logre darse a conocer o no logre ser de suficiente utilidad, el marco quedará completo cuando las tecnologías de la lista hayan sido añadidas. En caso de que el marco adquiera uso y la demanda aumente, es probable que la lista cambie de forma en función de esta demanda.

Así que, como último apartado de los últimos pasos, será añadir las siguientes tecnologías en el orden propuesto:
\begin{enumerate}
  \item Docker: es imprescindible que el marco de soporte a contenedores Docker y, por tanto, esta será la máxima prioridad.
  \item Enzyme: numerosos desarrolladores están trabajando con Enzyme en lugar de React Testing Library. Me parece requisito indispensable que acabe entrando dentro del marco como máxima prioridad.
  \item Redis: una base de datos libre que se utiliza en numerosos proyectos.
  \item Vue
  \item AngularJS
  \item Go: esta idea todavía no la tengo clara, dado que todo el marco está en NodeJS. Sin embargo, sería adecuado si otros lenguajes pudiesen ser utilizados en el lado de servidor. Go parece ser la alternativa a NodeJS que más está entrando en el mercado. De todos modos, esta opción queda pendiente a ser evaluada cuando llegue el momento.
\end{enumerate}
