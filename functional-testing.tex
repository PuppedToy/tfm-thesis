Las pruebas funcionales o \gls{e2e}, son aquellas pruebas que son totalmente agnósticas de la tecnología utilizada para desarrollar la aplicación. Se encargan de interactuar con ella como si fuesen el usuario final. En el caso de las aplicaciones web, se suelen utilizar navegadores de tipo Headless para realizar estas pruebas de forma automática. Estas pruebas son totalmente independientes de React y NodeJS, aunque lo ideal es desarrollarlas en el mismo lenguaje que el resto de pruebas que se realizan en la aplicación.

Para decidir qué herramienta de pruebas \gls{e2e} se va a utilizar, se han considerado las citadas en el artículo \cite{MPSFT19}. Dentro de estas, se han descartado todas ellas que no tienen compatibilidad con NodeJS para el desarrollo de los casos de prueba o aquellas que no tiene una versión gratuita. Así que se han tenido en cuenta únicamente Selenium y Crypress.

Tal y como explica \citet{CYPVSEL}, Selenium es una herramienta madura y estable, con mucha funcionalidad y que tiene configuración algo más pesada. Sin embargo, Selenium permite grabar las acciones que se realizan sobre el navegador y generar los casos de prueba automáticamente. Y esto es exactamente lo que interesa al marco de trabajo que se propone: ser de configuración rápida y fácil uso para una aplicación pequeña. De la configuración se encargará el propio marco, así que la capacidad de grabar el navegador es una gran ventaja con respecto a Cypress. Además, Selenium está preparado para trabajar con Jest.

Por otro lado, Cypress está específicamente diseñado para entornos de NodeJS y está creciendo a gran velocidad. Como sugiere \citet{CYPVSEL}, Cypress debería considerarse como una alternativa a futuro, que según vaya creciendo se puede ir incorporando a este proyecto. Hoy por hoy, Selenium es líder en herramientas de pruebas \gls{e2e} y, con todo el apoyo que da a entornos de Javascript y Node, no hay ninguna otra que sea apropiada para la tarea.
