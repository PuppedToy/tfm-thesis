Una vez ha sido construido el marco de trabajo, se ha realizado el proyecto "TODO List" con él, tal y como se explica en el apartado \nameref{chap:development}. Este proyecto se ha realizado de dos formas distintas: primero se ha realizado sin el marco de trabajo, y luego se ha repetido utilizando el nuevo marco.

El objetivo de este experimento es comprobar cuánto esfuerzo supone realizar un proyecto muy pequeño de forma que el resultado tenga calidad y sea escalable y, de ese modo, comprobar desde qué tipo de proyecto puede empezar a merecer la pena aplicar dicho esfuerzo.

La motivación original de este proyecto era conseguir reducir este esfuerzo inicial que supone realizar este trabajo mantenible y escalable. Lo ideal para este experimento habría sido que hubiese una población más grande y dividirla en tres grupos de desarrollo:

\begin{enumerate}
\item Los que hacen el proyecto sin aportar calidad.
\item Los que hacen el proyecto a partir del marco de trabajo y sin conocerlo previamente.
\item Los que hacen el proyecto con calidad, pero sin utilizar el marco de trabajo.
\end{enumerate}

De este modo, podríamos realizar un verdadero estudio estadísitico sobre lo que aporta de verdad el marco. Sin embargo, dados los recursos de este trabajo (tanto de tiempo como de capacidad de trabajo), tengo que reducir el experimento a un solo desarrollador (yo mismo) y prescindir del tercer grupo de trabajo. Además, una vez hecho el proyecto con marco de trabajo, al realizar el proyecto sin marco de trabajo ya habré pasado por un proceso previo de diseño y habré aprendido de mis errores en el desarrollo anterior. Por todo esto, la valided del experimento debe ser tomada con cuidado.

Aun así, se pueden sacar conclusiones de este experimento. El objetivo es tener una idea general de cuánto supone este esfuerzo, hacer una estimación de costes para una empresa pequeña y, de paso, comprobar si he echado de menos algo del marco de trabajo para poder tenerlo en consideración a futuro. Para poder sacar el máximo partido al experimento, en este apartado he realizado predicción sobre los resultados que esperaba y por qué antes de realizar dicho experimento. Una vez hecho el experimento, he complementado las predicciones con comentarios sobre las sorpresas que he recibido y las conclusiones que he sacado de los números obtenidos.
