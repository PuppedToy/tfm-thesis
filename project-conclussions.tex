Durante este proyecto se ha investigado una numerosa cantidad de tecnologías y se han recogido las más frecuentes en el ámbito del desarrollo web. Se ha analizado cuáles combinan bien entre ellas, se ha creado un entorno que facilita la configuración rápida del proyecto y se ha generado un proyecto de pruebas para cada tecnología seleccionada.

El objetivo era facilitar la producción de pruebas de concepto de calidad, de modo que pudiesen escalar de forma rápida y sin errores. Para eso hace falta seguir una serie de buenas prácticas y construir una buena batería de pruebas desde el primer minuto.

Para demostrar que este objetivo se cumple, se ha realizado un experimento sencillo: se ha creado un proyecto en una versión sin marco y otra versión con marco. Se han comparado, y ha resultado que la versión con marco, para este caso particular, nos ha ahorrado 20 errores en tan solo 4 horas más de codificación. Si consideramos el tiempo de resolver esos errores, el marco nos ha permitido dedicar el mismo tiempo que si lo hubiésemos hecho sin marco, pero teniendo la batería de pruebas generada y, por tanto, previniendo futuros errores que pudiesen surgir con la evolución del proyecto.

Después de todo el esfuerzo y viéndolo con perspectiva, he llegado a la conclusión de que la cantidad de tecnologías soportadas no es suficiente para cubrir la mayoría de necesidades del mundo del desarrollo web y, cuantas más tecnologías cubra el marco, más rápida va a ser la configuración de las pruebas de concepto y más útil va a ser el marco. Así que, para cubrir un mayor número de tecnologías, hace falta abrir el código y trabajar más en el producto.

Aun así, la experiencia de utilizar el marco de trabajo tal y como está ahora ha sido muy gratificante. Tener un esqueleto de proyecto que asegura calidad tan solo respondiendo unas cuantas preguntas, que ese esqueleto sea flexible a las tecnologías implicadas y que incluya ficheros de configuración para integración continua es exactamente lo que personalmente necesito para motivarme a que mis pruebas de concepto tengan calidad desde el primer momento.

Así que espero ser capaz de transmitir esta experiencia a otros usuarios y que poco a poco este marco se vaya transformando en una necesidad para todo aquél que desee empezar un proyecto de cero. También espero que el marco sea semilla de proyectos de calidad y, de este modo, animar al resto de desarrolladores a comenzar con buen pie cada idea que tengan por pequeña que sea.
