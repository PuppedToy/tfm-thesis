La mayoría de las veces un proyecto comienza como una prueba de concepto. Escribiendo algunas líneas aquí y allá para ver finalmente si funciona, ver cómo la audiencia lo recibe y actuar según la retroalimentación. Esto permite un enfoque de bajo riesgo para un beneficio potencial. Sin embargo, este código generalmente se realiza a un ritmo rápido con poca o ninguna calidad. Podemos entender el código de calidad como cualquier código que siga las buenas prácticas, sea mantenible, se pruebe automáticamente y aproveche la integración y la implementación continuas. En resumen, es un código que es menos propenso a errores y se escala fácilmente.

Esta prueba de concepto puede tener éxito y si no se tiene en cuenta la calidad, también puede terminar en un código de crecimiento muy caro. La calidad no es opcional en productos medianos o grandes, ya que permite mantener un buen ritmo con las actualizaciones con la confianza de que casi no tendrá errores y será fácil de depurar.

La mayoría de las veces, establecer un buen entorno para el código de calidad significa duplicar o incluso triplicar el esfuerzo para productos pequeños, como pruebas de concepto. Por lo tanto, también es común ver que estos pequeños proyectos de investigación se creen sin buenas prácticas ni pruebas. Si el experimento es un fracaso, no pasa nada. Sin embargo, si el resultado es un éxito, es aún más costoso reconstruir la prueba de concepto en un producto escalable, comprobable y automatizado.

En este trabajo voy a tratar de aliviar un poco la carga para que la configuración de un entorno de código de calidad sea más fácil para un proyecto pequeño. El objetivo es crear un marco de trabajo que pueda aplicar automáticamente las buenas prácticas, crear un flujo de trabajo de integración continua y un esqueleto de pruebas para un proyecto para ayudar con este difícil proceso y hacer que los experimentos sean más baratos sin arriesgar la escalabilidad de nuestro exitoso experimento. Para demostrar que la producción de código de calidad vale la pena incluso en los proyectos más pequeños, voy a probar este nuevo marco en un ejemplo y medir los tiempos y los errores, para que podamos verificar como de costoso es codificar bien desde el principio.
