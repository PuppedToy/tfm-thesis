La mayoría de las veces un proyecto comienza como una prueba de concepto. Escribiendo algunas líneas aquí y allá para ver finalmente si funciona, ver cómo la audiencia lo recibe y actuar según la retroalimentación. Esto permite un enfoque de bajo riesgo para un beneficio potencial. Sin embargo, este código generalmente se realiza a un ritmo rápido con poca o ninguna calidad. Podemos entender el código de calidad como cualquier código que siga las buenas prácticas, sea mantenible, se pruebe automáticamente y aproveche la integración y la implementación continuas. En resumen, es un código que es menos propenso a errores y se escala fácilmente.

Esta prueba de concepto puede tener éxito y si no se tiene en cuenta la calidad, también puede terminar en un código de crecimiento muy caro. La calidad no es opcional en productos medianos o grandes, ya que permite mantener un buen ritmo con las actualizaciones con la confianza de que casi no tendrá errores y será fácil de depurar.

En este trabajo trato de aliviar un poco la carga para que la configuración de un entorno de código de calidad sea más fácil para un proyecto pequeño. Se ha creado un marco de trabajo que pueda aplicar automáticamente las buenas prácticas, crear un flujo de trabajo de integración continua y un esqueleto de pruebas para un proyecto para ayudar con este difícil proceso y hacer que los experimentos sean más baratos sin arriesgar la escalabilidad de nuestro exitoso experimento.

Para realizar el marco he centrado el objetivo en el desarrollo web y he hecho un estudio de mercado para conocer las tecnologías más utilizadas. He creado un esqueleto de código para cada tecnología elegida y he creado un ejecutable que, mediante unas preguntas, prepara el proyecto con el esqueleto adecuado para las necesidades del desarrollador. Además, el marco se encarga de proporcionar las guías necesarias y ficheros de configuración para proporcionar integración contínua desde el principio.

Por último, he demostrado que utilizando el marco se puede crear un proyecto pequeño con código de calidad en el mismo tiempo en el que se realiza el mismo proyecto sin calidad, siempre y cuando se incluya el tiempo de resolución de errores que genera un proyecto sin código probado automáticamente.
