El alcance del trabajo está limitado a las horas impuestas y, dadas las circunstancias, se ha decidido acotar a unas pocas tecnologías elegidas generalmente mediante el criterio de la popularidad, de forma que este marco pueda ser útil al mayor número posible de desarrolladores. El código ya ha sido escrito con eso en mente y, por tanto, se ha preparado el terreno para dar un paso evidente en el ciclo: publicar el código como abierto.

Esto tiene las siguientes ventajas:
\begin{enumerate}
  \item El código es revisado por numerosos desarrolladores, dando distintos puntos de vista y aportando diversidad.
  \item El código se produce a más velocidad y los errores se resuelven más rápido, sin coste alguno.
  \item El proyecto se puede reutilizar y se pueden generar otros marcos y otras ideas.
  \item El marco se puede ampliar a más tecnologías porque más usuarios lo van a dar forma en función de sus necesidades.
\end{enumerate}

Sin embargo, el código abierto no es un paso pequeño. Tal y como constata el estándar de código abierto \cite{OPENSRC}, tiene una serie de responsabilidades para el dueño originario del código para que pueda funcionar, que en este marco son necesarias. Estas responsabilidades son, en general, buenas prácticas para cualquier código, pero en este marco en particular son unos requisitos indispensables a ser abordados antes de poder publicar el código como abierto:
\begin{enumerate}
  \item Todo el código debe estar revisado y comentado en función de una guía de buenas prácticas. La dificultad de esta tarea se va a reducir porque se utilizó Linter desde el primer momento, pero aun así una revisión de calidad es necesaria.
  \item El marco debe estar documentado en su totalidad. En el estado actual del proyecto, toda la documentación se refiere a como construir un proyecto a partir del marco, pero el marco en sí debe ser documentado mediante el punto de vista del editor del marco. Esto implica documentar las tareas automáticas y una guía de amplicación de las tencologías disponibles. Para hacer estas guías, es razonable realizar una amplicación por mí mismo para comprobar que todo funciona como se había planteado.
  \item Elaborar los documentos requeridos para el código abierto: Licencia, README, Guía de contribución y Código de conducta. El README ya está generado, pero requerirá una revisión y quizá nuevos apartados para la instalación desde un punto de vista de contribución.
\end{enumerate}

En el momento en el que el código se transforme en código abierto, se habrá cumplido el verdadero propósito del proyecto y se considerará iniciada la primera versión oficial del producto. Por tanto, el código ha de estar totalmente completo desde todos los puntos de vista posibles. El estado actual del producto es que es completo y listo para ser utilizado como marco, como semilla de otros proyectos. Pero faltan una serie de pasos para poder ser ampliado libremente y esto requiere planificación y revisión, así que esto será, en definitiva, el próximo paso más inmediato.
