\textbf{INTRODUCTION}: Proofs of concept meant to be quick usually become big scaled products in no time. If quality is not taken care of from the beginning, the project can become hard to mantain and thus might fail or become too expensive.

\textbf{OBJECTIVES}: The primary objective is create a framework able to smoothen the load of creating a quality proof of concept, keeping good practices and automated tests from the beginning. This framework will be scoped at web development.

\textbf{METHODS}: The new framework was built to deal with the tedious setup tasks before the project starts, taking into account continuous integration, testing, good practices and being flexible with the technologies available. This framework has a skeleton prepared to ensure quality. To prove this framework useful, a project called \textit{TODO List} has been created using traditional methods and the new framework simultaneously, comparing the time that costed to build both of the project versions.

\textbf{RESULTS}: Traditional project was created faster than the new framework version, as testing was the most time-consuming task. However, traditional project included 20 errors that the new framework avoided due to all the testing done in it. Turns out that debugging and solving those errors made the traditional project be as time consuming as the new framework one. In the end, the new framework provided all the setup aid needed to boost the production speed to match projects without quality, while keeping a good tests battery and good practices.

\textbf{CONCLUSION}: It usually pays off to make quality code from the beginning and this new framework will help providing all the setup needed.
