Las pruebas de integración son aquellas pruebas que permiten comprobar de forma automática que un componente de la aplicación está funcionando bien cuando se utiliza junto con otro componente. Es posible que dos componentes funcionen correctamente cuando están aislados, pero en el momento en el que se unen ocurre funcionalidad inesperada. Las pruebas de integración permiten detectar este fenómeno para poder alertar al desarrollador. Si se desea conocer más sobre las pruebas de integración, recomiendo visitar \citet{DZITWH}.

Durante las pruebas unitarias, hemos hablado de dos librerías de pruebas en React: Enzyme y React Testing Library. De hecho, gran parte de la discusión se ha centrado en un tema que, pese a que no tocaba en ese momento, era indivisible del resto de la comparación. El motivo por el que se ha elegido React Testing Library sobre Enzyme es porque es más preciso en las pruebas de integración. De hecho, Enzyme no realiza pruebas de integración dado que simula la generación de los hijos de un componente.

Así que en React las pruebas de integración y las de unidad están íntimamente relacionadas. Las unitarias se encargarían de comprobar la funcionalidad dentro de un componente React y las de integración, ver cómo se comportan los componentes hijos dentro de distintos componentes padre. En cualquier caso, tal y como nos dice \citet{RECTIT}, las librerías de pruebas unitarias y de integración son las mismas: Jest y React Testing Library.